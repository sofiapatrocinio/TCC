\chapter[Referencial Teórico]{Referencial Teórico}
\label{chap:Referencial}

Este capítulo apresenta o embasamento teórico dos conceitos chave abordados
neste trabalho, incluindo \hyperref[sec:Persona]{Persona}, \hyperref[sec:Usabilidade]
{Usabilidade}, \hyperref[sec:Experiência de Usuário]{Experiência de Usuário} e 
\hyperref[sec:Teste de Comparação]{Teste de Comparação}. 
Serão elucidadas ainda técnicas e métricas relevantes na avaliação da usabilidade e 
da experiência de usuário. Por fim, têm-se o \hyperref[sec:Resumo do Capítulo]{Resumo do Capítulo}.

\section{Persona}
\label{sec:Persona}

O conceito de persona foi introduzido por Alan Cooper e caracterizado como 
arquétipos hipotéticos de usuários reais, definidos com rigor e precisão 
significativos \cite{cooper1999}. Ou seja, são representações fictícias 
baseadas em dados coletados de usuários em potencial, que oferecem uma visão 
mais clara das necessidades e preferências do público-alvo, e por consequência, 
suas reações antecipadas aos elementos de \textit{design} e cenários de uso - podendo 
assim, ajudar a criar produtos, os quais pessoas reais gostariam de utilizar 
\cite{pruitt2006}.

A criação de personas pode ser realizada por meio de diversas técnicas, que 
podem variar de acordo com o contexto e as restrições do projeto, como tempo e 
orçamento. É essencial que informações baseadas em pesquisas com usuários sejam 
usadas para criar personas precisas em sua concepção, a fim de evitar que falhem 
pelo não uso com rigor de dados e metodologias \cite{pruitt2006}.

As personas são resultado de uma investigação que analisa as características dos 
usuários e descreve seus perfis, sendo que apenas seus nomes e detalhes pessoais 
são fictícios. Quanto mais específicas forem as personas, mais eficazes elas serão 
como ferramentas de \textit{design} e comunicação, fornecendo uma compreensão mais profunda 
dos usuários e suas necessidades \cite{barbosa2010}.

O público alvo deste trabalho são pessoas indígenas e pessoas não indígenas que têm as línguas 
indígenas como área de interesse. Adicionalmente, serão definidas personas no intuito de criar uma 
identidade mais bem definida, ou seja, algo que sintetize as principais características dessas pessoas, 
usando como base personagens semifictícios orientados a dados e comportamentos reais. Afinal, segundo 
\citeonline{cooper1999}, lidar com uma definição muito abrangente, como é o caso de público alvo, tende a gerar 
simplificações e estereótipos não aderentes, por exemplo, à parte dos interessados. Como consequência, 
a solução - no caso do presente trabalho, produto de software - tende a não agradar parte do público alvo, 
o que não é desejado. Nesse sentido, dentre outros aspectos, é relevante ter em mente, quando se define uma 
persona, o valor do produto, buscando descrever pessoas com foco em dúvidas, preocupações, necessidades, anseios, 
desagrados, dentre outros aspectos \cite{barbosa2010}.

\begin{description}
  \item No intuito de diferenciar público alvo e persona, seguem exemplos elaborados pela autora deste trabalho:
        \begin{itemize}
            \item Público alvo: mulheres, entre 20 e 35 anos, que moram em grandes centros (ex. São Paulo, Rio de Janeiro, Brasília), 
            com renda mensal de até R\$ 7.000, e atuantes em áreas de TI, buscando maior capacitação em termos profissionais, e

            \item Persona: Ana, de 30 anos, é gerente de qualidade em uma \textit{startup}, com necessidade de ampliar seu conhecimento sobre 
            experiência de usuário para gerar produtos mais atrativos aos clientes. Sua rotina de trabalho é dura, sobrando pouco 
            tempo para se capacitar. Somado à falta de tempo, ainda ocorre jornada dupla, acumulando tarefas e precisando se dividir 
            entre ser gerente na \textit{startup}, ao mesmo tempo que a principal atuante em uma iniciativa de negócio próprio também na área de TI. 
            Não gosta de se envolver nos problemas financeiros da \textit{startup}, mas frequentemente fica triste ao ver colegas sendo demitidos em 
            função de produtos pouco lucrativos e que não atendem às expectativas dos clientes.
        \end{itemize}
\end{description}

\begin{figure}[h!]
	\centering
	\caption{Persona representada em \textit{Card}}
  \begin{adjustbox}{center}
	  \includegraphics[width=1.3\textwidth]{figuras/persona.eps}
  \end{adjustbox}
  \begin{tablenotes}[flushleft]
    \centering
    \item \textit{Fonte:} Autora.
  \end{tablenotes}
	\label{fig01}
\end{figure}

Personas costumam ser apresentadas em \textit{cards}, ou seja, usando uma representação mais criativa e ilustrada, conforme mostrado na 
Figura \ref{fig01}. Nesse caso, tem-se a Aline, 24 anos, estagiária, também da área de qualidade, atuante na equipe da gerente Ana. Admira 
muito o trabalho da Ana, e deseja ser como ela no futuro próximo. Adora conhecer pessoas, e está sempre disposta a ajudar. Fica 
triste quando um colega de equipe é demitido, e compartilha da mesma vontade de Ana de se capacitar cada vez mais em termos profissionais. 
Por estar estagiando, dispõe de mais tempo livre do que a Ana, tendo mais conforto para gerenciar seu tempo e se dedicar a novos aprendizados.

Adicionalmente, é possível estabelecer ainda uma antipersona. Segundo \citeonline{barbosa2010}, trata-se de uma representação, também usando 
personagem semi fictício, mas com o intuito de especificar alguém não desejado como usuário/cliente do produto, seja devido ao fato de ser 
alguém que pode fazer uso indevido do produto, impactando-o, por exemplo, de forma negativa; seja por ser alguém, ao qual o produto não se 
destina (i.e. não faz parte do público alvo). A Figura \ref{fig02} ilustra uma típica antipersona, que difere do público alvo estabelecido anteriormente, 
de mulheres entre 20 e 35 anos, atuantes na área de TI.

Como antipersona, tem-se o Júlio, 50 anos, atuante na área de \textit{marketing}, mas já em fim de carreira. Não deseja se atualizar, e procura realizar 
seu trabalho conforme já o faz tem muitos anos. Tem condição de vida estável, com plano de aposentadoria privado garantido. Nesse sentido, não se 
preocupa muito com novas capacitações profissionais. Seu tempo é dividido entre trabalho (meio período) e convívio com a família.

\begin{figure}[h!]
	\centering
	\caption{Antipersona representada em \textit{Card}}
  \begin{adjustbox}{center}
	  \includegraphics[width=1.1\textwidth]{figuras/antipersona.eps}
  \end{adjustbox}
  \begin{tablenotes}[flushleft]
    \centering
    \item \textit{Fonte:} Autora.
  \end{tablenotes}
	\label{fig02}
\end{figure}

\section{Usabilidade}
\label{sec:Usabilidade}

A ISO 9126-1 \cite{iso9126}, norma internacional que trata da qualidade do software, 
define usabilidade como a capacidade do produto de software de ser compreendido, 
aprendido, utilizado e atraente para o usuário, em condições específicas de uso. 
\begin{description}
    \item Essa norma define ainda cinco subcaracterísticas a serem avaliadas:
          \begin{itemize}
              \item Inteligibilidade: A facilidade de entendimento do software por parte dos usuários. Essa 
              subcaracterística avalia a clareza e a precisão das informações fornecidas pelo software, bem 
              como a facilidade de compreensão das funcionalidades disponíveis;

              \item Apreensibilidade: A capacidade de aprendizado do software por parte dos usuários. Avalia 
              o tempo que o usuário leva para aprender a usar o software de forma satisfatória, bem como a 
                facilidade de acesso às informações e ajudas necessárias para realização das tarefas;

              \item Operacionalidade: A facilidade de uso e controle do software pelo usuário. Essa subcaracterística 
              avalia a facilidade de uso das funcionalidades do software, incluindo a clareza e a facilidade de uso 
              da interface do usuário; facilidade de acesso às funcionalidades; personalização, e facilidade de 
              correção de erros;

              \item Atratividade: Refere-se à aparência visual e à atratividade do software. Avalia a qualidade da 
              apresentação visual do software, incluindo a estética da interface; a legibilidade do texto; a clareza 
              dos ícones e gráficos, e a adequação do estilo visual ao público-alvo, e

              \item Conformidade: Diz respeito à conformidade do software com as convenções e os padrões estabelecidos para 
              a plataforma em que é executado. Essa subcaracterística avalia a conformidade do software com as normas, 
              convenções, guias de estilo ou regulamentações relacionadas à usabilidade.
          \end{itemize}
\end{description}


\citeonline{nielsen1994usability} define o critério de usabilidade como um conjunto de fatores que qualificam 
quão bem o usuário pode interagir com um sistema interativo, e propõe um conjunto de dez 
heurísticas de usabilidade para \textit{design} de interfaces de usuário.

\begin{description}
  \item Seguem as heurísticas com breves colocações:
        \begin{itemize}
            \item Visibilidade do status do sistema: O sistema deve manter os usuários informados sobre o que está 
            acontecendo, por meio de \textit{feedback} apropriado em tempo hábil. Como exemplo, pode ser mencionado 
            o uso de \textit{pop-ups} informativos que evidenciam quanto tempo falta para o completo \textit{download} de um arquivo;

            \item Compatibilidade do sistema com o mundo real: O sistema deve falar a linguagem dos usuários, com 
            palavras, frases e conceitos familiares para eles, em vez de termos orientados ao sistema. Como exemplo, pode ser 
            mencionado o uso de palavras e frases comuns em uma aplicação como "traduzir", "pronúncia", "sinônimos" e "antônimos", 
            ao invés de termos e jargões técnicos como "transliteração", "transcrição fonética", "termos sinônimos" e "termos antônimos";

            \item Controle do usuário e liberdade: Os usuários devem ter a opção de desfazer ações indesejadas, ou sair 
            de uma ação atual sem ter que passar por um longo processo. O recurso \textit{undo} pode ser aplicado em diversas ações, como 
            desfazer exclusão de um email e recuperar informações importantes;

            \item Consistência e padrões: Os usuários não devem ter que se perguntar se diferentes palavras, situações 
            ou ações significam a mesma coisa. Em vez disso, o sistema deve seguir padrões reconhecidos e consistentes. Pode-se mencionar o uso 
            de ícones já reconhecidos por usuários, como o de alfinete para representar mapa ou localização;

            \item Prevenção de erros: O sistema deve ser projetado para evitar erros sempre que possível, por meio de 
            mensagens de confirmação, alertas e limites para as ações dos usuários. Como por exemplo, pode ser mencionado 
            o bloqueio de botão de submeter antes do usuário finalizar o preenchimento de um formulário;
            
            \item Reconhecimento em vez de memorização: As informações, opções e ações importantes devem ser visíveis e 
            facilmente acessíveis, em vez de escondidas ou exigirem que os usuários se lembrem delas. Como por exemplo, pode ser mencionado 
            o uso de barras de menu superior em aplicativos de texto com opções frequentemente utilizadas como "abrir novo arquivo", "salvar", e "criar 
            arquivo novo";

            \item Flexibilidade e eficiência de uso: O sistema deve fornecer atalhos, recursos para usuários avançados e 
            opções para personalização, a fim de melhorar a eficiência e a flexibilidade do uso. Um exemplo é o uso de atalhos personalizáveis, que 
            permitem que os usuários configurem combinações de teclas para acessar rapidamente funções específicas;

            \item Estética e \textit{design} minimalista: O \textit{design} da interface deve ser estético, simples e consistente para ajudar
             os usuários a focar no conteúdo e nas tarefas. A página do Google é um excelente exemplo de estética e \textit{design} minimalista, com aparência simples, 
             uso de cores básicas e poucas fontes, facilitando assim que o usuário se concentre na tarefa em questão, sem distrações desnecessárias; 
            
            \item Ajuda aos usuários a reconhecer, diagnosticar e corrigir erros: O sistema deve ajudar os usuários a entender e corrigir erros por meio de mensagens 
            de erro claras, sugestões e instruções. Como exemplo, pode ser mencionado o uso de validadores em campos de email e telefone, os quais, em caso de erro, 
            mostram uma mensagem de erro que ajuda o usuário a identificar e corrigir o erro, e

            \item Ajuda e documentação: O sistema deve fornecer ajuda e documentação para os usuários quando necessário, incluindo 
            documentação de suporte, tutoriais e FAQ (Perguntas Frequentes) para solucionar problemas comuns. Um exemplo de FAQ de um site conhecido é o do site da 
            Amazon, que fornece uma página de perguntas frequentes, acompanhadas por respostas detalhadas e concisas, ajudando o usuário a resolver o problema ou encontrar 
            determinada informação.
        \end{itemize}
\end{description}

Já a ISO 9241-11 \cite{iso9241}, norma que trata da usabilidade em sistemas interativos, define usabilidade como sendo o grau em que 
um produto é usado por usuários específicos, para atingir objetivos particulares com eficácia, eficiência e satisfação em um dado contexto 
de uso.

Segundo essa norma, eficácia está relacionada com a capacidade dos usuários interagirem com o sistema para alcançarem seus objetivos 
corretamente, de forma satisfatória. A eficiência refere-se aos recursos necessários para os usuários interagirem 
com o sistema e alcançarem seus objetivos como tempo, esforço e materiais envolvidos. Já a satisfação é definida pelo grau de satisfação e 
conforto do usuário durante o uso do sistema interativo \cite{iso9241}.

A avaliação dessas métricas é importante para garantir que o sistema seja eficiente e eficaz, a fim de que os usuários atinjam seus objetivos 
com menor número de erros e com menor uso de recursos, levando a uma melhor satisfação do usuário em relação ao sistema e sentimentos de domínio 
e autoconfiança \cite{nielsen1994usability}.

\subsection{Medição}
\label{sec:Medição1}

O teste de usabilidade visa avaliar a usabilidade de um sistema interativo a partir de experiências de uso dos seus usuários-alvo \cite{rubin2011}. 
Para conduzir as medições desejadas, um grupo de usuários é convidado a realizar um conjunto de tarefas usando o sistema em um ambiente controlado. Durante a experiência de 
uso, são registrados dados sobre o desempenho dos participantes na realização das tarefas, podendo medir, por exemplo, o grau de sucesso da execução de tarefas; o total de 
erros cometidos; o tempo de execução para executá-las, e o número de ajudas necessárias \cite{barbosa2010}.

O objetivo do teste de usabilidade é fornecer informações por meio da coleta de dados para detectar e corrigir falhas de usabilidade e assegurar que o sistema seja valorizado 
e útil para o público-alvo; fácil de aprender; apoiador, em termos de eficácia e eficiência, nas tarefas que as pessoas desejam executar, e satisfatório de usar \cite{rubin2011}. 

Algumas atividades devem ser seguidas para realizar o teste de usabilidade, como pode ser visto no Tabela \ref{tab01}.

\begin{table}[h]
    \centering
    \caption{Teste de Usabilidade}
    \label{tab01}
    \begin{tabular}{l|l}
    \hline
    \textbf{Atividade}                   &  \textbf{Tarefa}                                                                                       \\ \hline
    \multirow{4}{*}{Preparação} &
      \begin{tabular}[c]{@{}l@{}}Definir tarefas para os participantes executarem\end{tabular} \\ \cline{2-2} 
                                & \begin{tabular}[c]{@{}l@{}}Definir o perfil dos participantes e recrutá-los\end{tabular}    \\ \cline{2-2} 
                                & \begin{tabular}[c]{@{}l@{}}Preparar material para observar e registrar o uso\end{tabular} \\ \cline{2-2} 
                                & Executar um teste-piloto                                                                     \\ \hline
    Coleta de Dados &
      \begin{tabular}[c]{@{}l@{}}Observar e registrar o desempenho e a opinião \\ dos participantes durante sessões de uso \\controladas\end{tabular} \\ \hline
    Interpretação &
      \multirow{2}{*}{\begin{tabular}[c]{@{}l@{}}Reunir, contabilizar e sumarizar os dados \\coletados dos participantes\end{tabular}} \\ \cline{1-1}
    Consolidação dos Resultados &                                                                                              \\ \hline
    Relato dos Resultados       & Relatar o desempenho e a opinião dos participantes                                          \\ \hline
    \end{tabular}
    \begin{tablenotes}[flushleft]
      \centering
      \item \textit{Fonte:} \cite{barbosa2010}.
    \end{tablenotes}
\end{table}

\section{Experiência de Usuário}
\label{sec:Experiência de Usuário}

O termo \textit{User Experience}\footnote{The Definition of User Experience (UX). Portal NN/g Nielsen Norman Group, 2017. Disponível
em: \url{https://www.nngroup.com/articles/definition-user-experience/} (último acesso: Julho 2023)} - em português, Experiência de Usuário - foi popularizado por Donald Norman, 
que o define, junto a Jakob Nielsen, como a totalidade da experiência de alguém ao interagir com um produto, serviço ou sistema. 

Assim como Norman, a ISO 9241-219 \cite{iso9241210} define a implementação de um processo de \textit{design} centrado no usuário, e caracteriza experiência de usuário como percepções e 
respostas de uma pessoa, resultantes do uso (antes, durante e após) de um sistema, produto ou serviço. Além disso, a norma enfatiza que a experiência de usuário 
é influenciada por diversos fatores, tais como: funcionalidade; desempenho do sistema; comportamento interativo, e capacidades assistivas. Além disso, tem-se foco em habilidades; 
personalidade; experiências anteriores, e contexto de uso do usuário.

Os autores \citeonline{hassenzahl2006} propõem ainda que a experiência de usuário é uma consequência do estado interno de um usuário; das características e dos fatores do sistema projetado, 
e do contexto em que a interação ocorre. Quando se deseja avaliar a experiência de usuário, deve-se fazer uso de mecanismos de medição apropriados, conforme colocado na próxima seção.

\subsection{Medição}
\label{sec:Medição2}

A fim de assegurar uma experiência de usuário satisfatória em um produto final, ou de aprimorá-la ao longo do ciclo de desenvolvimento, é fundamental realizar avaliações confiáveis e válidas. 
O \textit{AttrakDiff} é uma das ferramentas de avaliação disponíveis, projetada para avaliar a opinião do usuário, considerando suas próprias experiências, qualidade e usabilidade de sistemas \cite{hassenzahl2003}. 

\begin{description}
\item O questionário \textit{AttrakDiff} conta com 28 itens que avaliam um produto em relação às quatro dimensões principais:
      \begin{itemize}
          \item Qualidade Hedônica-Estímulo (QHE): Indica a quantidade de suporte que o produto oferece ao usuário para se 
          desenvolver, estimular e aumentar a motivação;

          \item Qualidade Hedônica-Identidade (QHI): Mede o grau de identificação da aplicação com o usuário;

          \item Qualidade Pragmática (QPR): Avalia a qualidade de uma aplicação e mede o grau de sucesso e esforço que 
          os usuários têm ao utilizá-la para atingir seus objetivos, e

          \item Atratividade (ATT): Mede a classificação global baseada nas qualidades percebidas.
      \end{itemize}
\end{description}

As avaliações são realizadas baseadas em uma escala com sete pontos, que contém adjetivos opostos, para indicar qual 
representa de forma mais adequada a experiência do usuário em que nível \cite{natashatayana2015} \cite{hassenzahl2003}.

\begin{description}
    \item A análise desses resultados, de acordo com \citeonline{providencia2021}, dá-se por meio de três formas: 
          \begin{itemize}
              \item Descrição de pares de palavras: Mostra os valores médios de cada par de palavras, sendo essas agrupadas em quatro dimensões;
    
              \item Portfólio dos resultados: Composto por quadrantes, fazendo uso da análise da Qualidade Pragmática e da Qualidade 
              Hedônica sob a perspectiva de média, e
    
              \item Diagrama de valores médios: Apresenta a média das quatro dimensões do produto sob as quatro dimensões.
          \end{itemize}
\end{description}

\citeonline{providencia2021} pontuam que, em uma aplicação do AttrakDiff, foi detectada uma insatisfação por parte dos respondentes devido ao tamanho 
do questionário. Perante isto, desenvolveram uma adaptação da ferramenta, o AttrakDiff Reduzido (AttrakDiff-R), utilizando 18 pares de palavras e a mesma 
escala de sete pontos Likert, como pode ser visto na Tabela \ref{tab02}.

\begin{table}[h]
    \centering
    \caption{Pares de palavras do Questionário AttrakDiff}
    \label{tab02}
    \begin{tabular}{c|ll}
    \hline
    \textbf{Dimensão}             & \multicolumn{2}{c}{\textbf{Par de Palavras}}                                 \\ \hline
    \multirow{5}{*}{QPR} & \multicolumn{1}{l|}{Técnico}                 & Humano                \\ \cline{2-3} 
                         & \multicolumn{1}{l|}{Complicado}              & Simples               \\ \cline{2-3} 
                         & \multicolumn{1}{l|}{Imprevisível}            & Previsível            \\ \cline{2-3} 
                         & \multicolumn{1}{l|}{Confuso}                 & Bem Estruturado       \\ \cline{2-3} 
                         & \multicolumn{1}{l|}{Incontrolável}           & Gerenciável           \\ \hline
    \multirow{4}{*}{QHE} & \multicolumn{1}{l|}{Sem imaginação}          & Criativo              \\ \cline{2-3} 
                         & \multicolumn{1}{l|}{Cauteloso}               & Ousado                \\ \cline{2-3} 
                         & \multicolumn{1}{l|}{Entediante}              & Chamativo             \\ \cline{2-3} 
                         & \multicolumn{1}{l|}{Pouco Exigente}          & Desafiador            \\ \hline
    \multirow{5}{*}{QHI} & \multicolumn{1}{l|}{Não profissional}        & Profissional          \\ \cline{2-3} 
                         & \multicolumn{1}{l|}{Não apresentável}        & Apresentável          \\ \cline{2-3} 
                         & \multicolumn{1}{l|}{De baixa qualidade}      & De alta qualidade     \\ \cline{2-3} 
                         & \multicolumn{1}{l|}{Alienador}               & Integrador            \\ \cline{2-3} 
                         & \multicolumn{1}{l|}{Me aproxima das pessoas} & Me afasta das pessoas \\ \hline
    \multirow{4}{*}{ATT} & \multicolumn{1}{l|}{Decepcionado}            & Realizado             \\ \cline{2-3} 
                         & \multicolumn{1}{l|}{Feio}                    & Bonito                \\ \cline{2-3} 
                         & \multicolumn{1}{l|}{Mau}                     & Bom                   \\ \cline{2-3} 
                         & \multicolumn{1}{l|}{Desencorajador}          & Motivador             \\ \hline
    \end{tabular}
    \begin{tablenotes}[flushleft]
      \centering
      \item \textit{Fonte:} \cite{hassenzahl2003}.
    \end{tablenotes}
\end{table}

O tipo de avaliação a ser empregado utilizando a ferramenta pode variar de acordo com o contexto, sendo: (i) Avaliação Única, aplicando-se o questionário uma única vez para avaliar 
a experiência do usuário, (ii) Teste A/B, avaliando-se duas versões diferentes do mesmo sistema, e (iii) Comparação Antes e Depois, aplicando-se o AttrakDiff antes e 
depois de uma mudança ou melhoria no sistema,  visando compreender melhor sobre a eficácia da mudança \cite{nzongo2018}.

\section{Teste de Comparação}
\label{sec:Teste de Comparação}

O teste de comparação é empregado para comparar dois ou mais \textit{designs}, que podem ser duas opções de interface diferentes; ou o \textit{design} atual de um sistema com um novo 
\textit{design} proposto, ou ainda, para comparar um produto com o de um concorrente. O teste de comparação é frequentemente utilizado para determinar qual produto/projeto é mais fácil 
de utilizar ou aprender; ou para obter uma compreensão mais precisa das vantagens e desvantagens de diferentes produtos/projetos. Dessa forma, esse tipo de teste é pertinente para aprimorar 
o processo de \textit{design} e tornar os produtos mais eficientes e satisfatórios para os usuários \cite{rubin2011}.

Em testes comparativos, os mesmos usuários podem tentar realizar tarefas em todos os produtos (\textit{within-subjects design}); ou diferentes grupos de usuários trabalham em cada 
produto (\textit{between-subjects design}), sendo preferível que cada participante realize testes em todas as condições \cite{lewis2016}. 

Os testes de comparação aderentes ao contexto do trabalho referem-se aos testes de comparação aplicáveis à Engenharia de Software. Nesse sentido, segundo \citeonline{pressman2021}, quando um 
software é desenvolvido, etapas são seguidas, resultando no produto desejado. Nesse processo, deseja-se qualidade. Portanto, o ciclo de vida de desenvolvimento de software atua como 
um \textit{framework}, sendo suas atividades gerenciadas por equipes especializadas \cite{pressman2021}. Como é desejado mitigar problemas e maximizar a qualidade e a satisfação do usuário, testes são de 
suma relevância \cite{delamaro2013}. Se testes forem combinados com outras estratégias da área de qualidade, podem ser obtidos resultados ainda mais interessantes \cite{damo2020} \cite{cintra2022} \cite{akinyemi2020}.

Diante do exposto, e já procurando acordar uma abordagem combinada de testes e planejamento estratégio, mapear forças, fraquezas, oportunidades e ameaças tende a proporcionar maior conhecimento 
sobre a solução proposta, em especial, no que tange a comparação com uma solução anterior ou concorrente \cite{damo2020} \cite{akinyemi2020}. Forças, Fraquezas, Oportunidades e Ameaças remetem à Matriz de SWOT \cite{fernandes2015}, 
sendo essa uma ferramenta de planejamento estratégico, comumente utilizada na análise de cenários - incluindo os de mercado, comportando-se como uma aliada na tomada de decisões. 

Conforme enfatizado por \citeonline{rubin2011}, testes de comparação são adequados para identificar vantagens e desvantagens em produtos e processos. Sendo assim, pode-se imaginar que esses testes também sejam pertinentes para 
identificar forças e fraquezas, instigando uma equipe ou um desenvolvedor em particular a compreender, respectivamente, oportunidades e ameaças na solução proposta. Esses testes tornam-se, portanto, 
aliados na evolução constante de soluções de software. Adicionalmente, cabe ter em mente que qualquer parte do software é candidata ao teste de comparação, tal como: a própria interface do usuário; 
uma dada funcionalidade de relevância; a camada de persistência, ou ainda outros aspectos em particular (ex. segurança, confiabilidade, privacidade, dentre outros).

Essa liberdade na escolha do que será testado também é algo que corrobora para a escolha dos testes de comparação. Ressalta-se ainda que os testes de comparação podem ser realizados em qualquer etapa 
do ciclo de vida de um software, além de poderem ocorrer de forma individual ou em conjunto com outros testes de software. Um exemplo bem sucedido de uso de testes de comparação, o qual permitiu identificar 
forças e fragilidades, é o estudo promovido por \citeonline{govardhan2010}, em que compara cinco modelos de Engenharia de Software, reportando os resultados obtidos. 

No caso do presente trabalho, pretende-se usar testes de comparação para comparar a solução melhorada do aplicativo Multilind, orientando-se pela aplicação das heurísticas de Nielsen, com a solução inicial 
do aplicativo. O foco da comparação será a experiência de usuário, e o método para aferir essa experiência será o questionário AttrakDiff. Pretende-se documentar os resultados obtidos a partir do questionário, 
procurando destacar: Forças, Fraquezas, Oportunidades e Ameaças. Ao identificar forcas e fragilidades, sob a perspectiva da experiência de usuário, espera-se reconhecer e evidenciar, respectivamente, oportunidades 
e ameaças. Tal noção quanto ao comportamento da solução proposta - no caso, o aplicativo melhorado - permitirá não apenas responder a Questão de Pesquisa inerente ao trabalho, como também acredita-se que será 
possível orientar trabalhos futuros para evolução do aplicativo Multilind, mantendo-o mais aderente às expectativas do público alvo. Por fim, espera-se que essa experiência, uma vez documentada, auxilie terceiros 
na compreensão sobre os principais impactos da usabilidade na experiência de usuário com apresentação de insumos.


\section{Resumo do Capítulo}
\label{sec:Resumo do Capítulo}

Este capítulo expôs o embasamento teórico de quatro conceitos fundamentais para o presente trabalho: Persona, Usabilidade, Experiência de Usuário e Teste de Comparação. No que se 
refere ao conceito de Persona, são apresentados os principais aspectos da sua definição, bem como sua importância para representar o público-alvo e suas necessidades de forma mais adequada, 
com uso de personagens semifictícios, a fim de criar produtos que atendam às expectativas dos usuários. 

Em relação à Usabilidade, são detalhadas as cinco sub características a serem avaliadas, conforme a norma ISO 9126-1, que incluem inteligibilidade, apreensibilidade, operacionalidade, 
atratividade e conformidade. Além disso, são apresentadas as dez heurísticas propostas por Nielsen para \textit{design} de interfaces de usuário. 

Por fim, são abordados os conceitos de Experiência de Usuário e Teste de Comparação, bem como aspectos fundamentais na avaliação de usabilidade e satisfação do usuário com o produto, 
destacando-se a importância de técnicas e métricas relevantes para a avaliação desses aspectos, e procurando apoiar-se em técnicas da área de Qualidade - Matriz de SWOT - na identificação de 
forças e fragilidades para permitir evoluções contínuas da solução, orientando-se por oportunidades e mitigando ameaças.