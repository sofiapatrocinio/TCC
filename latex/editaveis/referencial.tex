\chapter[Referencial Teórico]{Referencial Teórico}

Este capítulo apresenta o embasamento teórico dos conceitos chaves abordados
neste trabalho, incluindo Persona, Usabilidade e Experiência de Usuário. Serão
elucidados os principais conceitos e definições de cada um deles, além de 
técnicas e métricas para medir indicadores relevantes na avaliação da 
usabilidade e experiência de usuário. Por fim, tem-se as considerações finais.

\section{Persona}
\label{sec:Persona}

O conceito de persona foi introduzido por Alan Cooper e caracterizado como 
arquétipos hipotéticos de usuários reais, definidos com rigor e precisão 
significativos \cite{cooper1999}. Ou seja, são representações fictícias 
baseadas em dados coletados de usuários em potencial, que oferecem uma visão 
mais clara das necessidades e preferências do público-alvo, e por consequência, 
suas reações antecipadas aos elementos de design e cenários de uso - podendo 
assim, ajudar a criar produtos dos quais pessoas reais gostariam de utilizar 
\cite{pruitt2006}.

A criação de personas pode ser realizada por meio de diversas técnicas, que 
podem variar de acordo com o contexto e restrições do projeto, como tempo e 
orçamento. É essencial que informações baseadas em pesquisas com usuários sejam 
usadas para criar personas precisas em sua concepção, a fim de evitar que falhem 
pelo não uso com rigor de dados e metodologias \cite{pruitt2006}.

\section{Usabilidade}
\label{sec:Usabilidade}

A ISO 9126-1 \cite{iso9126}, norma internacional que trata da qualidade do software, 
define usabilidade como a capacidade do produto de software de ser compreendido, 
aprendido, utilizado e atraente para o usuário, em condições específicas de uso. 
\begin{description}
    \item Define, ainda, cinco subcaracterísticas a serem avaliadas:
          \begin{itemize}
              \item Inteligibilidade: A facilidade de entendimento do software por parte dos usuários. Essa 
              subcaracterística avalia a clareza e a precisão das informações fornecidas pelo software, bem 
              como a facilidade de compreensão das funcionalidades disponíveis;

              \item Apreensibilidade: A capacidade de aprendizado do software por parte dos usuários. Avalia 
              o tempo que o usuário leva para aprender a usar o software de forma satisfatória, bem como a 
                facilidade de acesso às informações e ajudas necessárias para realização das tarefas;

              \item Operacionalidade: A facilidade de uso e controle do software pelo usuário. Essa subcaracterística 
              avalia a facilidade de uso das funcionalidades do software, incluindo a clareza e a facilidade de uso 
              da interface do usuário, facilidade de acesso às funcionalidades, personalização e facilidade de 
              correção de erros;

              \item Atratividade: Refere-se à aparência visual e à atratividade do software. Avalia a qualidade da 
              apresentação visual do software, incluindo a estética da interface, a legibilidade do texto, a clareza 
              dos ícones e gráficos, e a adequação do estilo visual ao público-alvo.

              \item Conformidade: Diz respeito a conformidade do software com as convenções e padrões estabelecidos para 
              a plataforma em que é executado. Essa subcaracterística avalia a conformidade do software com as normas, 
              convenções, guias de estilo ou regulamentações relacionadas à usabilidade.
          \end{itemize}
\end{description}


\citeonline{nielsen1994usability} define o critério de usabilidade como um conjunto de fatores que qualificam 
quão bem o usuário pode interagir com um sistema interativo e propõe um conjunto de 10 
heurísticas de usabilidade para design de interfaces de usuário.

\begin{description}
  \item São elas:
        \begin{itemize}
            \item Visibilidade do status do sistema: O sistema deve manter os usuários informados sobre o que está 
            acontecendo, por meio de feedback apropriado em tempo hábil;

            \item Compatibilidade do sistema com o mundo real: O sistema deve falar a linguagem dos usuários, com 
            palavras, frases e conceitos familiares para eles, em vez de termos orientados ao sistema;

            \item Controle do usuário e liberdade: Os usuários devem ter a opção de desfazer ações indesejadas ou sair 
            de uma ação atual sem ter que passar por um longo processo.

            \item Consistência e padrões: Os usuários não devem ter que se perguntar se diferentes palavras, situações 
            ou ações significam a mesma coisa. Em vez disso, o sistema deve seguir padrões reconhecidos e consistentes.

            \item Prevenção de erros: O sistema deve ser projetado para evitar erros sempre que possível, por meio de 
            mensagens de confirmação, alertas e limites para as ações dos usuários.
            
            \item Reconhecimento em vez de memorização: As informações, opções e ações importantes devem ser visíveis e 
            facilmente acessíveis, em vez de escondidas ou exigirem que os usuários se lembrem delas.

            \item Flexibilidade e eficiência de uso: O sistema deve fornecer atalhos, recursos para usuários avançados e 
            opções para personalização, a fim de melhorar a eficiência e a flexibilidade do uso.

            \item Estética e design minimalista: O design da interface deve ser estético, simples e consistente para ajudar
             os usuários a focar no conteúdo e nas tarefas.
            
            \item Ajuda aos usuários a reconhecer, diagnosticar e corrigir erros: O sistema deve ajudar os usuários a entender 
            e corrigir erros por meio de mensagens de erro claras, sugestões e instruções.

            \item Ajuda e documentação: O sistema deve fornecer ajuda e documentação para os usuários quando necessário, incluindo 
            documentação de suporte, tutoriais e FAQ (Perguntas Frequentes) para solucionar problemas comuns.
        \end{itemize}
\end{description}

Já a ISO 9241-11 \cite{iso9241}, norma que trata da usabilidade em sistemas interativos, define usabilidade como sendo o grau em que 
um produto é usado por usuários específicos para atingir objetivos específicos com eficácia, eficiência e satisfação em um contexto 
de uso específico.

Segundo essa norma, eficácia está relacionada com a capacidade de os usuários interagirem com o sistema para alcançar seus objetivos 
corretamente, de forma satisfatória. A eficiência se refere aos recursos necessários para os usuários interagirem 
com o sistema e alcançarem seus objetivos como tempo, esforço e materiais envolvidos. Já a satisfação é definida pelo grau de satisfação e 
conforto do usuário durante o uso do sistema interativo.

A avaliação dessas métricas é importante para garantir que o sistema seja eficiente e eficaz, a fim de que os usuários atinjam seus objetivos 
com menor número de erros e com menor uso de recursos, pois caso exija muito tempo e esforço dos usuários para realizar simples tarefas pode 
ser considerado ineficiente e afetar negativamente a experiência do usuário.

\section{Experiência de Usuário}
\label{sec:Experiência de Usuário}


