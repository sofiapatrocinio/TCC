\chapter[Referencial Teórico]{Referencial Teórico}

Este capítulo apresenta o embasamento teórico dos conceitos chaves abordados
neste trabalho, incluindo Persona, Usabilidade e Experiência de Usuário. Serão
elucidados os principais conceitos e definições de cada um deles, além de 
técnicas e métricas para medir indicadores relevantes na avaliação da 
usabilidade e experiência de usuário. Por fim, tem-se as considerações finais.

\section{Persona}
\label{sec:Persona}

O conceito de persona foi introduzido por Alan Cooper e caracterizado como 
arquétipos hipotéticos de usuários reais, definidos com rigor e precisão 
significativos \cite{cooper1999}. Ou seja, são representações fictícias 
baseadas em dados coletados de usuários em potencial, que oferecem uma visão 
mais clara das necessidades e preferências do público-alvo, e por consequência, 
suas reações antecipadas aos elementos de design e cenários de uso - podendo 
assim, ajudar a criar produtos dos quais pessoas reais gostariam de utilizar 
\cite{pruitt2006}.

A criação de personas pode ser realizada por meio de diversas técnicas, que 
podem variar de acordo com o contexto e restrições do projeto, como tempo e 
orçamento. É essencial que informações baseadas em pesquisas com usuários sejam 
usadas para criar personas precisas em sua concepção, a fim de evitar que falhem 
pelo não uso com rigor de dados e metodologias \cite{pruitt2006}.