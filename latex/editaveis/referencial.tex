\chapter[Referencial Teórico]{Referencial Teórico}

Este capítulo apresenta o embasamento teórico dos conceitos chave abordados
neste trabalho, incluindo \hyperref[sec:Persona]{Persona}, \hyperref[sec:Usabilidade]
{Usabilidade}, \hyperref[sec:Experiência de Usuário]{Experiência de Usuário} e 
\hyperref[sec:Teste de Comparação]{Teste de Comparação}. 
Serão elucidadas ainda técnicas e métricas relevantes na avaliação da usabilidade e 
da experiência de usuário. Por fim, têm-se as \hyperref[sec:Considerações Finais]{Considerações Finais}.

\section{Persona}
\label{sec:Persona}

O conceito de persona foi introduzido por Alan Cooper e caracterizado como 
arquétipos hipotéticos de usuários reais, definidos com rigor e precisão 
significativos \cite{cooper1999}. Ou seja, são representações fictícias 
baseadas em dados coletados de usuários em potencial, que oferecem uma visão 
mais clara das necessidades e preferências do público-alvo, e por consequência, 
suas reações antecipadas aos elementos de \textit{design} e cenários de uso - podendo 
assim, ajudar a criar produtos, os quais pessoas reais gostariam de utilizar 
\cite{pruitt2006}.

A criação de personas pode ser realizada por meio de diversas técnicas, que 
podem variar de acordo com o contexto e às restrições do projeto, como tempo e 
orçamento. É essencial que informações baseadas em pesquisas com usuários sejam 
usadas para criar personas precisas em sua concepção, a fim de evitar que falhem 
pelo não uso com rigor de dados e metodologias \cite{pruitt2006}.

As personas são resultado de uma investigação que analisa as características dos 
usuários e descreve seus perfis, sendo que apenas seus nomes e detalhes pessoais 
são fictícios. Quanto mais específicas forem as personas, mais eficazes elas serão 
como ferramentas de \textit{design} e comunicação, fornecendo uma compreensão mais profunda 
dos usuários e suas necessidades. \cite{barbosa2010}

\section{Usabilidade}
\label{sec:Usabilidade}

A ISO 9126-1 \cite{iso9126}, norma internacional que trata da qualidade do software, 
define usabilidade como a capacidade do produto de software de ser compreendido, 
aprendido, utilizado e atraente para o usuário, em condições específicas de uso. 
\begin{description}
    \item Essa norma define ainda cinco subcaracterísticas a serem avaliadas:
          \begin{itemize}
              \item Inteligibilidade: A facilidade de entendimento do software por parte dos usuários. Essa 
              subcaracterística avalia a clareza e a precisão das informações fornecidas pelo software, bem 
              como a facilidade de compreensão das funcionalidades disponíveis;

              \item Apreensibilidade: A capacidade de aprendizado do software por parte dos usuários. Avalia 
              o tempo que o usuário leva para aprender a usar o software de forma satisfatória, bem como a 
                facilidade de acesso às informações e ajudas necessárias para realização das tarefas;

              \item Operacionalidade: A facilidade de uso e controle do software pelo usuário. Essa subcaracterística 
              avalia a facilidade de uso das funcionalidades do software, incluindo a clareza e a facilidade de uso 
              da interface do usuário, facilidade de acesso às funcionalidades, personalização e facilidade de 
              correção de erros;

              \item Atratividade: Refere-se à aparência visual e à atratividade do software. Avalia a qualidade da 
              apresentação visual do software, incluindo a estética da interface, a legibilidade do texto, a clareza 
              dos ícones e gráficos, e a adequação do estilo visual ao público-alvo.

              \item Conformidade: Diz respeito a conformidade do software com as convenções e padrões estabelecidos para 
              a plataforma em que é executado. Essa subcaracterística avalia a conformidade do software com as normas, 
              convenções, guias de estilo ou regulamentações relacionadas à usabilidade.
          \end{itemize}
\end{description}


\citeonline{nielsen1994usability} define o critério de usabilidade como um conjunto de fatores que qualificam 
quão bem o usuário pode interagir com um sistema interativo e propõe um conjunto de 10 
heurísticas de usabilidade para \textit{design} de interfaces de usuário.

\begin{description}
  \item São elas:
        \begin{itemize}
            \item Visibilidade do status do sistema: O sistema deve manter os usuários informados sobre o que está 
            acontecendo, por meio de feedback apropriado em tempo hábil;

            \item Compatibilidade do sistema com o mundo real: O sistema deve falar a linguagem dos usuários, com 
            palavras, frases e conceitos familiares para eles, em vez de termos orientados ao sistema;

            \item Controle do usuário e liberdade: Os usuários devem ter a opção de desfazer ações indesejadas ou sair 
            de uma ação atual sem ter que passar por um longo processo.

            \item Consistência e padrões: Os usuários não devem ter que se perguntar se diferentes palavras, situações 
            ou ações significam a mesma coisa. Em vez disso, o sistema deve seguir padrões reconhecidos e consistentes.

            \item Prevenção de erros: O sistema deve ser projetado para evitar erros sempre que possível, por meio de 
            mensagens de confirmação, alertas e limites para as ações dos usuários.
            
            \item Reconhecimento em vez de memorização: As informações, opções e ações importantes devem ser visíveis e 
            facilmente acessíveis, em vez de escondidas ou exigirem que os usuários se lembrem delas.

            \item Flexibilidade e eficiência de uso: O sistema deve fornecer atalhos, recursos para usuários avançados e 
            opções para personalização, a fim de melhorar a eficiência e a flexibilidade do uso.

            \item Estética e \textit{design} minimalista: O \textit{design} da interface deve ser estético, simples e consistente para ajudar
             os usuários a focar no conteúdo e nas tarefas.
            
            \item Ajuda aos usuários a reconhecer, diagnosticar e corrigir erros: O sistema deve ajudar os usuários a entender 
            e corrigir erros por meio de mensagens de erro claras, sugestões e instruções.

            \item Ajuda e documentação: O sistema deve fornecer ajuda e documentação para os usuários quando necessário, incluindo 
            documentação de suporte, tutoriais e FAQ (Perguntas Frequentes) para solucionar problemas comuns.
        \end{itemize}
\end{description}

Já a ISO 9241-11 \cite{iso9241}, norma que trata da usabilidade em sistemas interativos, define usabilidade como sendo o grau em que 
um produto é usado por usuários específicos para atingir objetivos específicos com eficácia, eficiência e satisfação em um contexto 
de uso específico.

Segundo essa norma, eficácia está relacionada com a capacidade de os usuários interagirem com o sistema para alcançar seus objetivos 
corretamente, de forma satisfatória. A eficiência se refere aos recursos necessários para os usuários interagirem 
com o sistema e alcançarem seus objetivos como tempo, esforço e materiais envolvidos. Já a satisfação é definida pelo grau de satisfação e 
conforto do usuário durante o uso do sistema interativo \cite{iso9241}.

A avaliação dessas métricas é importante para garantir que o sistema seja eficiente e eficaz, a fim de que os usuários atinjam seus objetivos 
com menor número de erros e com menor uso de recursos, levando a uma melhor satisfação do usuário em relação ao sistema e sentimentos de domínio 
e autoconfiança \cite{nielsen1994usability}.

\subsection{Medição}
\label{sec:Medição1}

O teste de usabilidade visa avaliar a usabilidade de um sistema interativo a partir de experiências de uso dos seus usuários-alvo \cite{rubin2011}. 
Para realizar as medições desejadas, um grupo de usuários é convidado a realizar um conjunto de tarefas usando o sistema em um ambiente controlado. Durante a experiência de 
uso são registrados dados sobre o desempenho dos participantes na realização das tarefas, podendo medir, por exemplo, o grau de sucesso da execução de tarefas, o total de 
erros cometidos, tempo de execução para executá-las e número de ajudas necessárias \cite{barbosa2010}.

O objetivo do teste de usabilidade é fornecer informações por meio da coleta de dados para detectar e corrigir falhas de usabilidade e assegurar que os sistema seja valorizado 
e útil para o público-alvo, fácil de aprender, ajude a pessoas a serem eficazes e eficientes nas tarefas que desejam executar e seja satisfatório de usar \cite{rubin2011}. 

Algumas atividades devem ser seguidas para realizar o teste de usabilidade, como pode ser visto na tabela \ref{tab01}.

% Please add the following required packages to your document preamble:
% \usepackage{multirow}
\begin{table}[h]
    \centering
    \caption{Teste de Usabilidade \protect\cite{barbosa2010}}
    \label{tab01}
    \begin{tabular}{|l|l|}
    \hline
    Atividade                   & Tarefa                                                                                       \\ \hline
    \multirow{4}{*}{Preparação} &
      \begin{tabular}[c]{@{}l@{}}Definir tarefas para os participantes executarem\end{tabular} \\ \cline{2-2} 
                                & \begin{tabular}[c]{@{}l@{}}Definir o perfil dos participantes e recrutá-los\end{tabular}    \\ \cline{2-2} 
                                & \begin{tabular}[c]{@{}l@{}}Preparar material para observar e registrar o uso\end{tabular} \\ \cline{2-2} 
                                & Executar um teste-piloto                                                                     \\ \hline
    Coleta de Dados &
      \begin{tabular}[c]{@{}l@{}}Observar e registrar a performance e a opinião \\ dos participantes durante sessões de uso \\controladas\end{tabular} \\ \hline
    Interpretação &
      \multirow{2}{*}{\begin{tabular}[c]{@{}l@{}}Reunir, contabilizar e sumarizar os dados \\coletados dos participantes\end{tabular}} \\ \cline{1-1}
    Consolidação dos Resultados &                                                                                              \\ \hline
    Relato dos Resultados       & Relatar a performance e a opinião dos participantes                                          \\ \hline
    \end{tabular}
\end{table}

\section{Experiência de Usuário}
\label{sec:Experiência de Usuário}

O termo \textit{User Experience} foi popularizado por Donald Norman, que o define, junto a Jakob Nielsen, como a totalidade da experiência de alguém ao interagir com um produto, 
serviço ou sistema. \footnote{The Definition of User Experience (UX). Portal NN/g Nielsen Norman Group, 2017. Disponível
em: \url{https://www.nngroup.com/articles/definition-user-experience/} (último acesso: Abril 2023)}

Assim como Norman, a ISO 9241-219 \cite{iso9241210} define a implementação de um processo de \textit{design} centrado no usuário e caracteriza experiência de usuário como percepções e 
respostas de uma pessoa resultantes do uso (antes, durante e após) de um sistema, produto ou serviço. Além disso, a norma enfatiza que a experiência de usuário 
é influenciada por diversos fatores como funcionalidade, desempenho do sistema, comportamento interativo, capacidades assistivas. Bem como habilidades, 
personalidade, experiências anteriores e contexto de uso do usuário.

Os autores \citeonline{hassenzahl2006} propõem ainda que a experiência de usuário é uma consequência do estado interno de um usuário, das características e fatores do sistema projetado 
e do contexto em que a interação ocorre. 

\subsection{Medição}
\label{sec:Medição2}

A fim de assegurar uma experiência de usuário satisfatória em um produto final ou de aprimorá-la ao longo do ciclo de desenvolvimento, é fundamental realizar avaliações confiáveis e válidas. 
O AttrakDiff é uma das ferramentas de avaliação disponíveis, projetada para avaliar a opinião do usuário em relação às suas experiências, à qualidade e à usabilidade de sistemas, levando em 
conta a perspectiva do usuário e de potenciais usuários. 

\begin{description}
\item O questionário conta com 28 itens que avaliam um produto em relação a quatro dimensões principais:
      \begin{itemize}
          \item Qualidade Hedônica-Estímulo (QHE): Indica a quantidade de suporte que o produto dá ao usuário para se 
          desenvolver, estimular e aumentar a motivação;

          \item Qualidade Hedônica-Identidade (QHI): Mede o grau de identificação da aplicação com o usuário;

          \item Qualidade Pragmática (QPR): Avalia a qualidade de uma aplicação e mede o grau de sucesso e esforço que 
          os usuários têm ao utilizá-la para atingir seus objetivos.

          \item Atratividade (ATT): Avalia a qualidade de uma aplicação e mede o grau de sucesso e esforço que os 
          usuários têm ao utilizá-la para atingir seus objetivos.
      \end{itemize}
\end{description}

As avaliações são realizadas baseadas em uma escala com sete pontos, que contém adjetivos opostos, para indicar qual  
melhor representa a experiência do usuário em que nível \cite{natashatayana2015} \cite{hassenzahl2003}.

\begin{description}
    \item A análise desses resultados se dá por meio de três formas: 
          \begin{itemize}
              \item Descrição de pares de palavras: Mostra os valores médios de cada par de palavras agrupadas em quatro dimensões;
    
              \item Portfólio dos resultados: Composto por quadrantes e utiliza a análise da Qualidade Pragmática e da Qualidade 
              Hedônica sob a perspectiva de média;
    
              \item Diagrama de valores médios: Apresenta a média das quatro dimensões do produto sob as quatro dimensões \cite{providencia2021}.
          \end{itemize}
\end{description}

\citeonline{providencia2021} pontuam que em uma aplicação do Attrakdiff foi detectada uma insatisfação por parte dos respondentes por conta do tamanho 
do questionário. Perante isto, desenvolveram uma adaptação da ferramenta, o Attrakdiff Reduzido (Attrakdiff-R) utilizando 18 pares de palavras e a mesma 
escala de sete pontos Likert, como pode ser visto na Tabela \ref{tab02}.

\begin{table}[h]
    \centering
    \caption{Pares de palavras do Questionário}
    \label{tab02}
    \begin{tabular}{|c|ll|}
    \hline
    Dimensão             & \multicolumn{2}{c|}{Par de Palavras}                                 \\ \hline
    \multirow{5}{*}{QPR} & \multicolumn{1}{l|}{Técnico}                 & Humano                \\ \cline{2-3} 
                         & \multicolumn{1}{l|}{Complicado}              & Simples               \\ \cline{2-3} 
                         & \multicolumn{1}{l|}{Imprevisível}            & Previsível            \\ \cline{2-3} 
                         & \multicolumn{1}{l|}{Confuso}                 & Bem Estruturado       \\ \cline{2-3} 
                         & \multicolumn{1}{l|}{Incontrolável}           & Gerenciável           \\ \hline
    \multirow{4}{*}{QHE} & \multicolumn{1}{l|}{Sem imaginação}          & Criativo              \\ \cline{2-3} 
                         & \multicolumn{1}{l|}{Cauteloso}               & Ousado                \\ \cline{2-3} 
                         & \multicolumn{1}{l|}{Entediante}              & Chamativo             \\ \cline{2-3} 
                         & \multicolumn{1}{l|}{Pouco Exigente}          & Desafiador            \\ \hline
    \multirow{5}{*}{QHI} & \multicolumn{1}{l|}{Não profissional}        & Profissional          \\ \cline{2-3} 
                         & \multicolumn{1}{l|}{Não apresentável}        & Apresentável          \\ \cline{2-3} 
                         & \multicolumn{1}{l|}{De baixa qualidade}      & De alta qualidade     \\ \cline{2-3} 
                         & \multicolumn{1}{l|}{Alienador}               & Integrador            \\ \cline{2-3} 
                         & \multicolumn{1}{l|}{Me aproxima das pessoas} & Me afasta das pessoas \\ \hline
    \multirow{4}{*}{ATT} & \multicolumn{1}{l|}{Decepcionado}            & Realizado             \\ \cline{2-3} 
                         & \multicolumn{1}{l|}{Feio}                    & Bonito                \\ \cline{2-3} 
                         & \multicolumn{1}{l|}{Mau}                     & Bom                   \\ \cline{2-3} 
                         & \multicolumn{1}{l|}{Desencorajador}          & Motivador             \\ \hline
    \end{tabular}
\end{table}

O tipo de avaliação a ser empregado utilizando a ferramenta pode variar de acordo com o contexto, sendo eles Avaliação Única, aplicado uma única vez para avaliar 
a experiência do usuário, Teste A/B, para avaliar duas versões diferentes do mesmo sistema e a Comparação Antes e Depois, em que é aplicado o AttrakDiff antes e 
depois de uma mudança ou melhoria no sistema, para avaliar a eficácia da mudança \cite{nzongo2018}.

\section{Teste de Comparação}
\label{sec:Teste de Comparação}

O teste de comparação é empregado para comparar dois ou mais \textit{designs}, que podem ser duas opções de interface diferentes ou o \textit{design} atual de um sistema com um novo 
\textit{design} proposto, ou ainda, para comparar um produto com o de um concorrente. O teste de comparação é frequentemente utilizado para determinar qual projeto é mais fácil 
de utilizar ou aprender, ou para obter uma compreensão mais precisa das vantagens e desvantagens de diferentes projetos. Dessa forma, esse tipo de teste é extremamente 
útil para aprimorar o processo de \textit{design} e tornar os produtos mais eficientes e satisfatórios para os usuários \cite{rubin2011}.

Em testes comparativos os mesmos usuários podem tentar realizar tarefas em todos os produtos (\textit{within-subjects design}) ou diferentes grupos de usuários trabalham em cada 
produto (\textit{between-subjects design}), sendo preferível que cada participante realize testes em todas as condições \cite{lewis2016}. 

\section{Considerações Finais}
\label{sec:Considerações Finais}

Este capítulo expôs o embasamento teórico de quatro conceitos fundamentais para o presente trabalho: Persona, Usabilidade, Experiência de Usuário e Teste de Comparação. No que se 
refere ao conceito de Persona, são apresentados os principais aspectos da sua definição, bem como sua importância para representar o público-alvo e suas necessidades, a fim de criar 
produtos que atendam às expectativas dos usuários. 

Em relação à Usabilidade, são detalhadas as cinco sub características a serem avaliadas, conforme a norma ISO 9126-1, que incluem inteligibilidade, apreensibilidade, operacionalidade, 
atratividade e conformidade. Além disso, são apresentadas as dez heurísticas propostas por Nielsen para design de interfaces de usuário. 

Por fim, são abordados os conceitos de Experiência de Usuário e Teste de Comparação e aspectos fundamentais na avaliação da usabilidade e satisfação do usuário com o produto, 
destacando-se a importância de técnicas e métricas relevantes para a avaliação desses aspectos.
