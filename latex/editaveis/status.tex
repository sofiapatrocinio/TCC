\chapter[Status do Trabalho]{\textit{Status} do Trabalho}
\label{chap:Status}
Com o objetivo de encerrar a primeira etapa deste trabalho, este capítulo apresentará o \textit{status} das \hyperref[sec:Atividades Realizadas]{Atividades
Realizada} até o momento. Além disso, serão retomados os \hyperref[sec:Objetivos Especificos Alcancados]{Objetivos Específicos Alcançados}. 
Por fim, será apresentada a \hyperref[sec:Proxima Etapa]{Próxima Etapa}, que confere um resumo a respeito das atividades da segunda etapa do presente trabalho.

\section{Atividades Realizadas}
\label{sec:Atividades Realizadas}
No contexto deste trabalho, a primeira etapa teve como objetivo principal desenvolver atividades que sustentassem a proposta 
de projeto para aprimorar a Experiência do Usuário e a Usabilidade de um aplicativo existente, o Multilind. A Tabela \ref{tab09} apresenta 
o progresso das atividades relacionadas a essa etapa inicial. É observado que todas as atividades planejadas foram concluídas com sucesso. 
Atualmente, apenas a atividade "Apresentar TCC 01'' está "A Concluir'', pois está prevista para ocorrer em breve, com a participação dos membros da banca 
examinadora.

\begin{table}[h!]
	\centering
	\caption{Atividades da Primeira Etapa do TCC}
	\label{tab09}
	\begin{tabularx}{\textwidth}{p{8cm}|p{2cm}|p{4cm}}
	\hline
    Atividade                                                        & \textit{Status}       & Artefato Produzido             \\ \hline
    Definir Tema                                                     & Concluído    & Presente Trabalho              \\
    Contextualizar Problema                                          & Concluído    & \hyperref[chap:Introducao]{Capítulo 1}                   \\
    Levantamento Bibliográfico                                       & Concluído    & \hyperref[chap:Referencial]{Capítulo 2}                     \\
    Definir Suporte Tecnológico                                      & Concluído    & \hyperref[chap:ReferencialTech]{Capítulo 3}                   \\
    Especificar Metodologia                                          & Concluído    & \hyperref[chap:Metodologia]{Capítulo 4}                     \\
    Coletar Informações à respeito de Usabilidade e UX do aplicativo & Concluído    & \hyperref[sec:Prova de Conceito]{Prova de Conceito} \\
    Desenvolver Protótipo baseado em Melhorias                       & Concluído    & \hyperref[sec:Melhorias Propostas]{Melhorias Propostas}         \\
    Determinar Proposta do Trabalho                                  & Concluído    & \hyperref[chap:Proposta]{Capítulo 5}                     \\
    Revisar TCC 1                                                    & Concluído    & -                              \\ 
    Apresentar TCC 1                                                 & A Concluir   & -                              \\ \hline
	\end{tabularx}
\end{table}

\section{Objetivos Específicos Alcançados}
\label{sec:Objetivos Especificos Alcancados}

Na introdução deste estudo, foi definido como \hyperref[sec:Objetivos]{Objetivo Geral}, a melhoria do aplicativo Multilind em termos de experiência do usuário e usabilidade. 
Para alcançá-lo, foram estabelecidos os \hyperref[sec:Objetivos]{Objetivos Específicos} a serem alcançados.

A Tabela \ref{tab10} apresenta a situação atual de cada um desses objetivos específicos estabelecidos. Apesar da - aparente - conclusão* de todos os Objetivos Específiicos 
estabelecidos, os objetivos "Condução da análise de resultados no aplicativo'' e "Documentação do trabalho'' foram cumpridos apenas para o escopo da primeira etapa do TCC. 
Para o pleno cumprimento desses objetivos, assim como para a adequada resposta à questão de pesquisa, ainda há necessidade de Condução da análise de resultados no aplicativo'' 
e "Documentação do trabalho'', mas para o escopo da segunda etapa do TCC. Além disso, até findar o trabalho como um todo, ainda podem ocorrer refinamentos em termos de referenciais: 
Teórico, Tecnológico e Metodológico. Tais refinamentos, por vezes, podem demandar novas versões, respectivamente, dos Capítulos \hyperref[chap:Referencial]{2}, 
\hyperref[chap:ReferencialTech]{3} e \hyperref[chap:Metodologia]{4}.

\begin{table}[h!]
	\centering
	\caption{Objetivos Específicos Alcançados}
	\label{tab10}
	\begin{tabularx}{\textwidth}{p{8cm}|p{2cm}|p{4cm}}
	\hline
    Objetivo                                                        & \textit{Status}       & Artefato Produzido             \\ \hline
    Especificação clara do público alvo                                                     & Concluído    & \hyperref[Publico-Alvo]{Público-Alvo}              \\
    Levantamento de referenciais teóricos                                          & Concluído    & \hyperref[chap:Referencial]{Capítulo 2}                   \\
    Detalhamento de tecnologias e outros recursos técnicos                                       & Concluído    & \hyperref[chap:ReferencialTech]{Capítulo 3}                    \\
    Uso de prototipação, visando validações pontuais junto aos interessados                                      & Concluído    & \hyperref[sec:Prova de Conceito]{Prova de Conceito}                   \\
    Condução da análise de resultados no aplicativo                                          & Concluído*    & \hyperref[sec:Segundo Ciclo]{Segundo Ciclo de Testes}                    \\
    Documentação do trabalho                       & Concluído*    & Presente Trabalho         \\ \hline
	\end{tabularx}
\end{table}

\section{Próxima Etapa}
\label{sec:Proxima Etapa}
A segunda etapa do TCC visa, principalmente, aplicar as melhorias a nível de desenvolvimento no aplicativo já existente, Multilind. Essas melhorias envolvem atualização de dependências e implementação 
de novas funcionalidades de acordo com o \hyperref[sec:Backlog de Melhorias]{\textit{Backlog} de Melhorias}. Além da validação incremental das histórias de usuário, permitindo que o \textit{feedback} 
dos \textit{stakeholders} seja considerado, e ajustes necessários sejam feitos ao longo do processo de desenvolvimento. Ressalta-se que esse processo de Análise de Resultados será orientado, de forma mais rigorosa, pela Metodolologia de Pesquisa-ação. Essa estabelece um protocolo com etapas bem definidas para conferir de forma adequada a exposição dos resultados obtidos.

Acredita-se, diante dos resultados muito positivos obtidos até o momento, que essa proposta será capaz de melhorar tanto a Usabilidade, quanto a Experiência de Usuário no aplicativo Multilind. 
A autora mantém-se, portanto, confiante.