\chapter[Referencial Tecnológico]{Referencial Tecnológico}

Este capítulo tem como propósito descrever o referencial tecnológico utilizado no 
desenvolvimento deste trabalho, abordando as ferramentas e tecnologias empregadas 
tanto na parte escrita quanto na parte prática, que engloba a prototipação, 
desenvolvimento e gerenciamento de versões. Ao longo deste capítulo, serão apresentadas 
as diversas categorias de ferramentas utilizadas, incluindo pesquisa e escrita, 
prototipação, avaliação e validação, e desenvolvimento.

\section{LaTeX}
\label{sec:Latex}
LaTeX \cite{latex} é uma ferramenta para escrita de documentos acadêmicos e científicos que oferece 
recursos de formatação e organização de texto. É baseada no sistema de composição tipográfica 
TeX, desenvolvido por Donald Knuth na década de 1970.

Esta monografia utiliza o LaTeX para a composição textual, além disso a referência bibliográfica 
é realizada de acordo com o padrão BibTeX, que facilita a formatação e organização das citações e 
referências bibliográficas em documentos escritos em LaTeX.

A fim de garantir uma experiência consistente e facilitar o processo de compilação do documento LaTeX, foi 
adotado o uso do Docker em conjunto com o Template TCC FGA-UnB, desenvolvido pelo professor 
Edson Alves. 

\section{Docker}
\label{sec:Docker}
Docker é uma plataforma \textit{open source} de virtualização que permite empacotar um ambiente completo em 
uma imagem, incluindo todas as dependências necessárias, garantindo que o projeto seja 
compilado de maneira consistente, independentemente do sistema operacional ou das configurações do computador 
utilizado.

\section{LaTeX Workshop}
\label{sec:LaTeX Workshop}
LaTeX Workshop é uma extensão para o Visual Studio Code útil para edição, compilação e visualização de 
documentos LaTeX diretamente no editor de texto Visual Studio Code. A ferramenta fornece compilação 
automática sempre que há alteração no documento, facilitando a visualização rápida das informações.

\section{Visual Studio Code}
\label{sec:Visual Studio Code}
Visual Studio Code é um editor de código-fonte gratuito de código aberto. Trata-se de um ambiente de desenvolvimento 
integrado (IDE) que oferece suporte a diversas linguagens de programação e tecnologias.

No contexto da monografia, é utilizado para escrita do próprio documento LaTeX, bem como para o desenvolvimento do código-fonte 
dos diferentes serviços da aplicação.

\section{Git e GiHub}
\label{sec:Git e GiHub}
Git é um sistema de controle de versão, que foi projetado para rastrear alterações em arquivos e diretórios. Através dele é 
possível manter um histórico de todas as alterações feitas em um repositório de código, permitindo acessar versões anteriores, 
comparar mudanças e trabalhar em paralelo em diferentes ramos (\textit{branches}).

Já o GitHub é uma plataforma de hospedagem de código-fonte baseada na web, que utiliza o Git como seu sistema de controle de 
versão. Permite a hospedagem, compartilhamento e colaboração entre desenvolvedores de um projeto de \textit{software}.

Cada edição deste documento e do código-fonte da aplicação Multilind foi versionada por meio do Git e armazenados pelo GitHub. 

\section{React Native}
\label{sec:React Native}
React Native é um framework baseado no React que permite o desenvolvimento de aplicativos móveis usando JavaScript. Uma das 
principais vantagens do React Native em relação a outros frameworks é a capacidade de converter o código desenvolvido para a 
linguagem nativa do sistema operacional, resultando em um desempenho mais fluido e eficiente. Com o React Native, é possível 
criar aplicativos tanto para Android quanto para iOS, aproveitando o mesmo código JavaScript.

Foi utilizado o React Native durante o desenvolvimento \textit{front-end} desta aplicação em conjunto com o Expo, que facilita 
o processo de desenvolvimento de aplicativos React Native e elimina a necessidade de configurações complexas, fornecendo ferramentas 
e serviços que facilitam a criação, teste e publicação de aplicativos multiplataforma.