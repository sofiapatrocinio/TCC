\chapter[Referencial Tecnológico]{Referencial Tecnológico}

Este capítulo tem como propósito descrever o referencial tecnológico utilizado no 
desenvolvimento deste trabalho, abordando as ferramentas e tecnologias empregadas 
tanto na parte escrita quanto na parte prática, que engloba a prototipação, 
desenvolvimento e gerenciamento de versões. Ao longo deste capítulo, serão apresentadas 
as diversas categorias de ferramentas utilizadas, incluindo pesquisa e escrita, 
prototipação, avaliação e validação, e desenvolvimento.

\section{LaTeX}
\label{sec:Latex}
LaTeX \cite{latex} é uma ferramenta para escrita de documentos acadêmicos e científicos que oferece 
recursos de formatação e organização de texto. É baseada no sistema de composição tipográfica 
TeX, desenvolvido por Donald Knuth na década de 1970.

Esta monografia utiliza o LaTeX para a composição textual, além disso a referência bibliográfica 
é realizada de acordo com o padrão BibTeX, que facilita a formatação e organização das citações e 
referências bibliográficas em documentos escritos em LaTeX.

A fim de garantir uma experiência consistente e facilitar o processo de compilação do documento LaTeX, foi 
adotado o uso do Docker em conjunto com o Template TCC FGA-UnB, desenvolvido pelo professor 
Edson Alves. 

\section{Docker}
\label{sec:Docker}
Docker é uma plataforma de virtualização que permite empacotar um ambiente completo em 
uma imagem, incluindo todas as dependências necessárias, garantindo que o projeto seja 
compilado de maneira consistente, independentemente do sistema operacional ou das configurações do computador 
utilizado.

\section{LaTeX Workshop}
\label{sec:LaTeX Workshop}
LaTeX Workshop é uma extensão para o Visual Studio Code útil para edição, compilação e visualização de 
documentos LaTeX diretamente no editor de texto Visual Studio Code. A ferramenta fornece compilação 
automática sempre que há alteração no documento, facilitando a visualização rápida das informações.

\section{Visual Studio Code}
\label{sec:Visual Studio Code}