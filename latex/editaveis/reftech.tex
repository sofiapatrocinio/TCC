\chapter[Referencial Tecnológico]{Referencial Tecnológico}

Este capítulo tem como propósito descrever o referencial tecnológico utilizado no 
desenvolvimento deste trabalho, abordando as ferramentas e tecnologias empregadas 
tanto na parte escrita quanto na parte prática, que engloba a prototipação, 
desenvolvimento e gerenciamento de versões. Ao longo deste capítulo, serão apresentadas 
as diversas categorias de ferramentas utilizadas, incluindo pesquisa e escrita, 
prototipação, avaliação e validação, e desenvolvimento.

\section{LaTeX}
\label{sec:Latex}
LaTeX \cite{latex} é uma ferramenta para escrita de documentos acadêmicos e científicos que oferece 
recursos de formatação e organização de texto. É baseada no sistema de composição tipográfica 
TeX, desenvolvido por Donald Knuth na década de 1970.

Esta monografia utiliza o LaTeX para a composição textual, além disso a referência bibliográfica 
é realizada de acordo com o padrão BibTeX, que facilita a formatação e organização das citações e 
referências bibliográficas em documentos escritos em LaTeX.

A fim de garantir uma experiência consistente e facilitar o processo de compilação do documento LaTeX, foi 
adotado o uso do Docker em conjunto com o Template TCC FGA-UnB, desenvolvido pelo professor 
Edson Alves. 

\section{Docker}
\label{sec:Docker}
Docker é uma plataforma \textit{open source} de virtualização que permite empacotar um ambiente completo em 
uma imagem, incluindo todas as dependências necessárias, garantindo que o projeto seja 
compilado de maneira consistente, independentemente do sistema operacional ou das configurações do computador 
utilizado.

\section{LaTeX Workshop}
\label{sec:LaTeX Workshop}
LaTeX Workshop é uma extensão para o Visual Studio Code útil para edição, compilação e visualização de 
documentos LaTeX diretamente no editor de texto Visual Studio Code. A ferramenta fornece compilação 
automática sempre que há alteração no documento, facilitando a visualização rápida das informações.

\section{Visual Studio Code}
\label{sec:Visual Studio Code}
Visual Studio Code é um editor de código-fonte gratuito de código aberto. Trata-se de um ambiente de desenvolvimento 
integrado (IDE) que oferece suporte a diversas linguagens de programação e tecnologias.

No contexto da monografia, é utilizado para escrita do próprio documento LaTeX, bem como para o desenvolvimento do código-fonte 
dos diferentes serviços da aplicação.

\section{Git e GiHub}
\label{sec:Git e GiHub}
Git é um sistema de controle de versão, que foi projetado para rastrear alterações em arquivos e diretórios. Através dele é 
possível manter um histórico de todas as alterações feitas em um repositório de código, permitindo acessar versões anteriores, 
comparar mudanças e trabalhar em paralelo em diferentes ramos (\textit{branches}).

Já o GitHub é uma plataforma de hospedagem de código-fonte baseada na web, que utiliza o Git como seu sistema de controle de 
versão. Permite a hospedagem, compartilhamento e colaboração entre desenvolvedores de um projeto de \textit{software}.

Cada edição deste documento e do código-fonte da aplicação Multilind foi versionada por meio do Git e armazenados pelo GitHub. 

\section{React Native}
\label{sec:React Native}
React Native é um framework baseado no React que permite o desenvolvimento de aplicativos móveis usando JavaScript. Uma das 
principais vantagens do React Native em relação a outros frameworks é a capacidade de converter o código desenvolvido para a 
linguagem nativa do sistema operacional, resultando em um desempenho mais fluido e eficiente. Com o React Native, é possível 
criar aplicativos tanto para Android quanto para iOS, aproveitando o mesmo código JavaScript.

Foi utilizado o React Native durante o desenvolvimento \textit{front-end} desta aplicação em conjunto com o Expo, que facilita 
o processo de desenvolvimento de aplicativos React Native e elimina a necessidade de configurações complexas, fornecendo ferramentas 
e serviços que facilitam a criação, teste e publicação de aplicativos multiplataforma.

\section{Expo}
\label{sec:Expo}
Expo é uma plataforma de código aberto que facilita a criação de aplicativos React Native. Possui ferramentas que  
possibilitam ter uma abordagem simplificada para criação e teste de aplicativos.

\begin{description}
    \item Entre as ferramentas que compõem o Expo:
          \begin{itemize}
              \item Expo SDK - conjunto de bibliotecas, APIs e recursos para o desenvolvimento de aplicativos móveis que permite acesso 
              à câmera, geolocalização, notificações push, armazenamento local etc;

              \item Expo Go - aplicativo disponível para iOS e Android que permite a visualização da aplicação em tempo real durante seu 
              desenvolvimemnto;

              \item Expo CLI - ferramenta de linha de comando que facilita a crição e configuração de projetos Expo, além de ferramentas para 
              testar, executar em ambiente de desenvolvimento e publicá-lo nas lojas de aplicativos.
          \end{itemize}
\end{description}

\section{Node.js}
\label{sec:Node.js}
Node.js é um ambiente de execução JavaScript de código aberto que permite executar JavaScript do lado do servidor. Possui como características 
a flexibilidade, escalabilidade e facilidade de lidar com solicitações simultâneas de forma assíncrona.

No contexto do projeto, está sendo utilizado para fazer o tratamento da base dos bancos de dados, e para criar 
APIs (Interfaces de Programação de Aplicativos) que são consumidas pelo aplicativo \textit{front-end}.

Em conjunto ao Node.js, o Express.js está sendo utilizado para simplificar o processo de desenvolvimento e tratamento de solicitações e respostas HTTP.

\section{Express.js}
\label{sec:Express.js}
Express.js é uma biblioteca para Node.js que simplifica o desenvolvimento de aplicativos web e APIs. Fornece uma estrutura leve e flexível para 
lidar com roteamento, gerenciamento de solicitações e respostas HTTP, middlewares etc.

Na aplicação Multilind, está sendo utilizada para criar APIs robustas e facéis de utilizar através da definição de dados a serem enviados e respostas 
no formato e código de status adequados. Além disso, o uso de middlewares para autenticação e validação de parâmetros das requisições também está presente.

\section{Figma}
\label{sec:Figma}
O Figma é uma ferramenta de design de interface de usuário (UI) utilizada para criar protótipos interativos e designs de alta fidelidade, de forma  
colaborativa em tempo real com outras pessoas. Essa ferramenta permite o compartilhamento de links de visualização interativos, simulando a experiência 
do usuário ao utilizar a aplicação de fato.

Toda a prototipação do aplicativo Multilind foi desenvolvida de forma colaborativa no Figma pelos membros da equipe de desenvolvimento, conforme pode ser visto 
na Figura X.

\section{Maze}
\label{sec:Maze}