\chapter[Referencial Tecnológico]{Referencial Tecnológico}
\label{chap:ReferencialTech}

Este capítulo tem como propósito descrever o referencial tecnológico utilizado no desenvolvimento
deste trabalho, abordando as ferramentas e tecnologias empregadas, com ênfase em: (i) \hyperref[sec:Apoio a Escrita]{Apoio à Escrita}; 
(ii) \hyperref[sec:Apoio à Prática]{Apoio à Prática}, que engloba \hyperref[sec:Prototipação]{Prototipação}, \hyperref[sec:Desenvolvimento]{Desenvolvimento}, \hyperref[sec:Gerenciamento de Versões]{Gerenciamento de Versões}, 
e \hyperref[sec:Análise de Resultados]{Análise de Resultados}, e (iii) \hyperref[sec:Apoio Complementar]{Apoio Complementar}. Por fim, tem-se o \hyperref[sec:Resumo do Capítulo]{Resumo do Capítulo}.

\section{Apoio à Escrita}
\label{sec:Apoio a Escrita}
Seguem as principais ferramentas utililizadas na elaboração dessa monografia, incluindo formatação do texto e compilação da documentação.

\subsection{LaTeX}
\label{sec:Latex}
LaTeX \cite{latex} é uma ferramenta para escrita de documentos acadêmicos e científicos que oferece 
recursos de formatação e organização de texto. É baseada no sistema de composição tipográfica 
TeX, desenvolvido por Donald Knuth na década de 1970.

Esta monografia utiliza o LaTeX para a composição textual. Além disso, a referência bibliográfica 
é realizada de acordo com o padrão BibTeX, que facilita a formatação e a organização das citações e 
referências bibliográficas em documentos escritos em LaTeX.

A fim de garantir uma experiência consistente e facilitar o processo de compilação do documento LaTeX, foi 
adotado o uso do Docker em conjunto com o Template TCC FGA-UnB, desenvolvido pelo professor 
Edson Alves\footnote{Template TCC FGA-UnB. Disponível
em: \url{https://github.com/fga-unb/template-latex-tcc} (último acesso: Julho 2023)}.

\subsection{Docker}
\label{sec:Docker}
Docker é uma plataforma \textit{open source} de virtualização que permite empacotar um ambiente completo em 
uma imagem, incluindo todas as dependências necessárias, garantindo que o projeto seja 
compilado de maneira consistente, independentemente do sistema operacional ou das configurações do computador   
utilizado \cite{docker}.

O Docker está sendo utilizado no contexto do LaTeX para garantir uma experiência consistente; facilitar o 
processo de compilação de documentos, e também no desenvolvimento \textit{backend} da aplicação. O intuito é evitar problemas 
de compatibilidade.

\subsection{LaTeX Workshop}
\label{sec:LaTeX Workshop}
LaTeX Workshop é uma extensão para o Visual Studio Code útil para edição, compilação, e visualização de 
documentos LaTeX diretamente no editor de texto Visual Studio Code. A ferramenta fornece compilação 
automática sempre que há alteração no documento, facilitando a visualização rápida das informações \cite{latexworkshop}.

\section{Apoio à Prática}
\label{sec:Apoio à Prática}
Seguem as principais ferramentas de apoio às atividades de Prototipação, Desenvolvimento, Gerenciamento de Versões, 
e Análise de Resultados.

\subsection{Prototipação}
\label{sec:Prototipação}
Será necessário prototipar novas telas, orientadas às melhorias no aplicativo, e visando planejamento e validações 
junto ao público alvo, antes mesmo do desenvolvimento dessas melhorias no aplicativo em si. Nesse contexto, deve-se 
usar o Figma.

\subsubsection{Figma}
\label{sec:Figma}
O Figma é uma ferramenta de \textit{design} de interface de usuário (UI) utilizada para criar protótipos interativos e \textit{designs} de alta fidelidade, de forma  
colaborativa em tempo real com outras pessoas. Essa ferramenta permite o compartilhamento de \textit{links} de visualização interativos, simulando a experiência 
do usuário ao utilizar a aplicação de fato \cite{figma}.

Toda a prototipação do aplicativo Multilind foi desenvolvida de forma colaborativa no Figma pelos membros da equipe de desenvolvimento, conforme pode ser visto 
na Figura \ref{fig03}. Nesse sentido, serão incorporadas melhorias nos protótipos, visando novas validações junto ao público alvo.  

\begin{figure}[h!]
	\centering
	\caption{Protótipo de Alta Fidelidade}
	\resizebox{1\textwidth}{!}{\includegraphics{figuras/prototype.eps}}
    \begin{tablenotes}[flushleft]
		\centering
		\item \textit{Fonte:} Autora.
	\end{tablenotes}
	\label{fig03}
\end{figure}

\subsection{Desenvolvimento}
\label{sec:Desenvolvimento}
O desenvolvimento das melhorias no aplicativo será orientado ao arcabouço tecnológico já utilizado na primeira versão do Multilind. Nesse sentido, seguem as 
principais tecnologias identificadas. Havendo a necessidade de alterações nessas escolhas, as mesmas serão devidamente justificadas.

\subsubsection{Visual Studio Code}
\label{sec:Visual Studio Code}
Visual Studio Code é um editor gratuito de código aberto. Trata-se de um ambiente de desenvolvimento 
integrado (IDE) que oferece suporte a diversas linguagens de programação e tecnologias \cite{vscode}.

No contexto da monografia, é utilizado para escrita do próprio documento LaTeX, bem como para o desenvolvimento do código-fonte 
dos diferentes serviços da aplicação.

\subsubsection{React Native}
\label{sec:React Native}
React Native é um \textit{framework} baseado no React que permite o desenvolvimento de aplicativos móveis usando JavaScript. Uma das 
principais vantagens do React Native em relação a outros \textit{frameworks} é a capacidade de converter o código desenvolvido para a 
linguagem nativa do sistema operacional, resultando em um desempenho mais fluido e eficiente. Com o React Native, é possível 
criar aplicativos tanto para Android quanto para iOS, aproveitando o mesmo código JavaScript \cite{reactnative}.

Foi utilizado o React Native durante o desenvolvimento \textit{front-end} desta aplicação em conjunto com o Expo, que facilita 
o processo de desenvolvimento de aplicativos React Native, e elimina a necessidade de configurações complexas, fornecendo ferramentas 
e serviços que facilitam a criação, o teste e a publicação de aplicativos multiplataforma.

\subsubsection{Expo}
\label{sec:Expo}
Expo é uma plataforma de código aberto que facilita a criação de aplicativos React Native. Possui ferramentas que  
possibilitam ter uma abordagem simplificada para criação e teste de aplicativos \cite{expo}.

\begin{description}
    \item Entre as ferramentas que compõem o Expo, merecem atenção:
          \begin{itemize}
              \item Expo SDK - conjunto de bibliotecas, APIs e recursos para o desenvolvimento de aplicativos móveis que permite acesso aos 
              recursos de câmera, geolocalização, notificações \textit{push}, armazenamento local, dentre outros;

              \item Expo Go - aplicativo disponível para iOS e Android que permite a visualização da aplicação em tempo real durante seu 
              desenvolvimento, e

              \item Expo CLI - ferramenta de linha de comando que facilita criação e configuração de projetos Expo, além de ferramentas para 
              testar; executar em ambiente de desenvolvimento, e publicar nas lojas de aplicativos.
          \end{itemize}
\end{description}

\subsubsection{Node.js}
\label{sec:Node.js}
Node.js é um ambiente de execução JavaScript de código aberto que permite executar JavaScript do lado do servidor. Possui como características 
flexibilidade, escalabilidade e facilidade de lidar com solicitações simultâneas de forma assíncrona \cite{nodejs}.

No contexto do projeto, está sendo utilizado para fazer o tratamento do banco dos bancos de dados, e para criar 
APIs (Interfaces de Programação de Aplicativos) que são consumidas pelo aplicativo \textit{front-end}.

Em conjunto ao Node.js, o Express.js está sendo utilizado para simplificar o processo de desenvolvimento e tratamento de solicitações e respostas HTTP.

\subsubsection{Express.js}
\label{sec:Express.js}
Express.js é uma biblioteca para Node.js que simplifica o desenvolvimento de aplicativos web e APIs. Fornece uma estrutura leve e flexível para 
lidar com roteamento, gerenciamento de solicitações e respostas HTTP, \textit{middlewares}, dentre outros \cite{expressjs}.

Na aplicação Multilind, está sendo útil para criar APIs robustas e fáceis de utilizar através da definição de dados a serem enviados, e respostas 
no formato e código de status adequados. Além disso, o uso de \textit{middlewares} para autenticação e validação de parâmetros das requisições também está presente.

\subsection{Gerenciamento de Versões}
\label{sec:Gerenciamento de Versões}
Ao longo do processo de melhorias, o versionamento torna-se necessário, visando obter um processo rastreável e de fácil apresentação sobre todas as evoluções 
proporcionadas. Além disso, há necessidade de manter o novo aplicativo hospedado em um repositório. Nesse contexto, faz-se uso do Git e GitHub.

\subsubsection{Git e GiHub}
\label{sec:Git e GiHub}
Git é um sistema de controle de versão, que foi projetado para rastrear alterações em arquivos e diretórios. Através dele, é 
possível manter um histórico de todas as alterações feitas em um repositório de código, permitindo acessar versões anteriores; 
comparar mudanças, e trabalhar em paralelo em diferentes ramos (\textit{branches}) \cite{git}.

Já o GitHub é uma plataforma de hospedagem de código-fonte baseada na web, que utiliza o Git como seu sistema de controle de 
versão. Permite hospedagem, compartilhamento e colaboração entre desenvolvedores de um projeto de \textit{software} \cite{github}.

Cada edição deste documento e do código-fonte da aplicação Multilind está sendo versionada por meio do Git e armazenada pelo GitHub. 

\subsection{Análise de Resultados}
\label{sec:Análise de Resultados}
Seja para validar os protótipos; ou as versões intermediárias obtidas; ou ainda a versão com as melhorias incorporadas, há necessidade de ferramentas específicas, 
tais como: Maze e Google Forms.

\subsubsection{Maze}
\label{sec:Maze}
Maze é uma ferramenta de teste de usabilidade que permite testar site, aplicativo ou protótipo com usuários reais. Por meio dela, é possível coletar \textit{feedback} sobre 
os \textit{designs} e garantir melhorias.

Além disso, o Maze oferece integração com o Figma, permitindo que o protótipo desenhado no Figma seja testado e validado no Maze. Dessa forma, permite que os usuários 
testem as funcionalidades a nível de protótipo. O Maze coleta os dados dos testes realizados pelos usuários e os disponibiliza, tornando possível a validação de usabilidade 
e experiência de usuário de uma determinada aplicação \cite{maze}.

\subsubsection{Google Forms}
\label{sec:Google Forms}
O Google Forms \cite{googleforms} é uma ferramenta \textit{online} fornecida pelo Google, que permite criar formulários personalizados para coletar informações de forma rápida e fácil. Neste estudo, foi 
utilizado para coletar dados dos \textit{stakeholders} a respeito da experiência de usuário na aplicação existente, sem a incorporação de melhorias.

De forma mais metodológica, será utilizado para realizar o questionário AttrakDiff e coletar respostas do público-alvo. Para realizar a coleta dessas informações, serão adicionados 
os antônimos em cada item do questionário com uma escala de Likert para indicar grau de concordância em relação às usabilidade e funcionalidade do produto.

\section{Apoio Complementar}
\label{sec:Apoio Complementar}
Seguem outros recursos que estão sendo consumidos para plena viabilização trabalho.

\subsection{Ferramenta de Comunicação}
\label{sec:Ferramenta de Comunicação}

\begin{description}
    \item Visando conferir facilidades na comunicação entre autora e orientadora, estão sendo utilizadas as seguintes ferramentas:
          \begin{itemize}
              \item Telegram\footnote{\url{https://web.telegram.org/}(último acesso: Julho 2023)} - para mensagens rápidas;

              \item Slack\footnote{\url{https://slack.com/}(último acesso: Julho 2023)} - para manter os rastros das orientações, e

              \item Teams\footnote{\url{https://teams.microsoft.com/}(último acesso: Julho 2023)} - para reuniões semanais.
          \end{itemize}
\end{description}

\subsection{Ferramenta de Modelagem}
\label{sec:Ferramenta de Modelagem}
Como há necessidade de modelar determinados aspectos, visando uma adequada apresentação de artefatos na monografia, faz-se uso da LucidChart, com base em notações como: UML e BPMN. Um exemplo de 
uso dessa ferramenta pode ser conferido no \hyperref[chap:Metodologia]{Capítulo 4 - Metodologia}, nas especificações dos processos inerentes às atividades a serem realizadas ao longo desse trabalho.

\subsubsection{LucidChart}
\label{sec:LucidChart}
O Lucidchart \cite{lucidchart} é uma plataforma focada em diagramas \textit{online} que permite criação, colaboração e compartilhamento de diagramas, além de atender várias notações, incluindo UML (\textit{Unified Modeling Language}) 
e BPMN (\textit{Business Process Model and Notation}).

\subsection{Gerenciamento do Fluxo de Trabalho}
\label{sec:Gerenciamento do Fluxo de Trabalho}
A autora conduz um processo de desenvolvimento solo, ou seja, sendo a única atuante no trabalho em termos, por exemplo, de programação. Assim, cabe o uso de uma ferramenta simples, intuitiva, e que 
permita o acompanhamento das atividades de forma clara e adequada. Nesse caso, tem-se o uso do Trello.

\subsubsection{Trello}
\label{sec:Trello}
O Trello \cite{trello} é uma plataforma de gerenciamento de projetos que utiliza quadros, listas e cartões para a organização de tarefas. Sua proposta é auxiliar indivíduos e equipes a organizar suas tarefas e acompanhar o progresso 
dos projetos de forma visual e colaborativa.

\section{Resumo do Capítulo}
\label{sec:Resumo do Capítulo 2}
Ao longo deste capítulo, foram apresentadas as diversas ferramentas e tecnologias empregadas para apoiar a elaboração deste trabalho, desde a pesquisa e escrita até o desenvolvimento. 
O Quadro \ref{tab03} apresenta um resumo dos principais aspectos discutidos ao longo deste capítulo.

\begin{table}[h]
    \centering
    \caption{Principais ferramentas do Referencial Tecnológico}
    \label{tab03}
    \resizebox{\columnwidth}{!}{\begin{tabular}{l|l|l|l}
        \hline
        \textbf{Nome}      & \textbf{Descrição}                                            & \textbf{Versão}                                                     & \textbf{\textit{Link}}                           \\ \hline
        LaTeX              & Ferramenta para escrita do documento                          & LaTeX2e                                                             & \href{https://www.latex-project.org/}{https://www.latex-project.org/} \\ \hline
        Docker             & Plataforma de virtualização de ambiente                       & 23.0.4                                                              & \href{https://www.docker.com/}{https://www.docker.com/}       \\ \hline
        Figma              & Ferramenta de \textit{design} de interface de usuário                  & 116.4.2                                                             & \href{https://www.figma.com/}{https://www.figma.com/}         \\ \hline
        Visual Studio Code & Editor de código-fonte e de escrita do documento              & 1.79.0                                                              & \href{https://code.visualstudio.com/}{https://code.visualstudio.com/} \\ \hline
        React Native       & \textit{Framework} que permite o desenvolvimento de aplicativos móveis & 0.63                                                                & \href{https://reactnative.dev/}{https://reactnative.dev/}       \\ \hline
        Node.js            & Ambiente de execução do desenvolvimento back-end da aplicação & v16                                                                 & \href{https://nodejs.org/}{https://nodejs.org/}            \\ \hline
        Git                & Versionamento de código-fonte e texto da monografia           & 2.25.1                                                              & \href{https://git-scm.com/}{https://git-scm.com/}           \\ \hline
        GitHub             & Hospedagem dos repositórios da aplicação e monografia         & -                                                                   & \href{https://github.com/}{https://github.com/}            \\ \hline
        Maze               & Ferramenta para aplicação dos testes de usabilidade           & -                                                                   & \href{https://maze.co/}{https://maze.co/}            \\ \hline
        Google Forms       & Ferramenta para criação de formulários                        & -                                                                   & \href{https://www.google.com/intl/pt-BR/forms/about/}{https://www.google.com/intl/pt-BR/forms/about/}            \\ \hline
        LucidChart         & Plataforma de criação de diagramas                            & -                                                                   & \href{https://lucidchart.com/pages/pt}{https://lucidchart.com/pages/pt}            \\ \hline
    \end{tabular}}
\end{table}