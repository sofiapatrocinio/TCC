\chapter[Proposta]{Proposta}
\label{chap:Proposta}
Neste capítulo, é retomado o contexto em que o trabalho pretende contribuir, apresentando a proposta do Trabalho de Conclusão de Curso. Inicialmente na seção de Contextualização é 
apresentado o domínio em que o estudo está inserido. Adicionalmente, com intuito de apresentar adequadamente a aplicação, a seção Detalhamento da Aplicação é exposta. A Prova de 
Conceito, com os ciclos de testes de usabilidade, na versão atual do aplicativo e na versão com melhorias. Por fim, são acordadas as Considerações Finais.

\section{Contextualização}
\label{sec:Contextualização}


\section{Prova de Conceito}
\label{sec:Prova de Conceito}
Com o objetivo de avaliar as melhorias propostas no aplicativo Multilind em relação a usabilidade e experiência de usuário, provas de conceito foram 
realizadas para avaliar tanto a última versão do aplicativo (v1.4.0), quanto a versão com melhorias propostas. O primeiro ciclo de testes foi realizado 
com objetivo de coletar métricas e informações de usabilidade e experiência do usuário. Já o segundo ciclo de testes visa validar as melhorias propostas e avaliar seus resultados.

Todos os testadores participaram voluntariamente, de acordo com os termos estabelecidos no Termo de Consentimento, que foi adaptado do 
modelo da Universidade de Araraquara (Universidade de Araraquara, 2023), disponível no Apêndice C para consulta.

\subsection{Primeiro Ciclo de Testes}
\label{sec:Primeiro Ciclo}
O primeiro ciclo de testes de usabilidade foi realizado com cinco testadores, que fazem parte do público alvo da aplicação. Foi realizado um teste abrangendo as sete funcionalidades 
principais do aplicativo, com o objetivo de identificar seus aspectos-chave. Durante o teste, os usuários forneceram \textit{feedback} instantâneo sobre sua percepção de cada funcionalidade testada. 

Além disso, como uma ferramenta adicional de teste de usabilidade, foi utilizado o Maze para avaliar a navegação e a facilidade de uso do aplicativo. Os testadores foram solicitados a realizar  
tarefas e o tempo necessário para concluir o percurso foi registrado. Essa abordagem permitiu identificar eventuais dificuldades de navegação e fornecer informações para aprimorar a usabilidade do aplicativo.
Após a conclusão dos testes, foi aplicado um questionário baseado no Formulário de \textit{Attrakdiff} para avaliar a usabilidade e a experiência do usuário no aplicativo Multilind.

\begin{description}
    \item As funcionalidades testadas foram:
	\begin{itemize}
		\item F01 - Visualizar línguas através do mapa
		\item F02 - Ver detalhes de uma língua ao clicar em um ponto no mapa
		\item F03 - Visualizar línguas por ordem alfabética
		\item F04 - Visualizar línguas por família linguística
		\item F05 - Ver dicionário de palavras de uma língua específica
		\item F06 - Ver tradução de uma palavra para o português formal
		\item F07 - Visualizar imagens relativas as palavras de uma língua
	\end{itemize}
\end{description}

\subsubsection{Resultado do Primeiro Ciclo de Testes}
\label{sec:Resultado do Primeiro Ciclo de Testes}
