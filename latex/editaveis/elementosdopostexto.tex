\chapter[Elementos do Pós-Texto]{Elementos do Pós-Texto}

Este capitulo apresenta instruções gerais sobre a elaboração e formatação dos 
elementos do pós-texto a serem apresentados em relatórios de Projeto de 
Graduação. São abordados aspectos relacionados a redação de referências 
bibliográficas, bibliografia, anexos e contra-capa.

\section{Referências Bibliográficas}

O primeiro elemento do pós-texto, inserido numa nova página, logo após o último 
capítulo do trabalho, consiste da lista das referencias bibliográficas citadas 
ao longo do texto.

Cada referência na lista deve ser justificada entre margens e redigida no 
formato Times New Roman com 11pts. Não é necessário introduzir uma linha em 
branco entre referências sucessivas.

A primeira linha de cada referencia deve ser alinhada a esquerda, com as demais 
linhas da referencia deslocadas de 0,5 cm a partir da margem esquerda. 

Todas as referências aparecendo na lista da seção \lq\lq Referências 
Bibliográficas\rq\rq\ devem estar citadas no texto. Da mesma forma o autor deve 
verificar que não há no corpo do texto citação a referências que por 
esquecimento não forma incluídas nesta seção.

As referências devem ser listadas em ordem alfabética, de acordo com o último 
nome do primeiro autor. Alguns exemplos de listagem de referencias são 
apresentados no Anexo I.

Artigos que ainda não tenham sido publicados, mesmo que tenham sido submetidos 
para publicação, não deverão ser citados. Artigos ainda não publicados mas que 
já tenham sido aceitos para publicação devem ser citados como \lq\lq in 
press\rq\rq.

A norma , que regulamenta toda a formatação a ser usada na 
elaboração de referências a diferente tipos de fontes de consulta, deve ser 
rigidamente observada. Sugere-se a consulta do trabalho realizado por 
, disponível na internet.

\section{Anexos}

As informações citadas ao longo do texto como \lq\lq Anexos\rq\rq\ devem ser 
apresentadas numa seção isolada ao término do trabalho, após a seção de 
referências bibliográficas. Os anexos devem ser numerados seqüencialmente em 
algarismos romanos maiúsculos (I, II, III, ...). A primeira página dos anexos 
deve apresentar um índice conforme modelo apresentado no Anexo I, descrevendo 
cada anexo e a página inicial do mesmo.

A referência explícita no texto à um dado anexo deve ser feita como 
\lq\lq Anexo 1\rq\rq. Referências implícitas a um dado anexo devem ser feitas 
entre parênteses como (Anexo I). Para referências a mais de um anexo as mesmas 
regras devem ser aplicadas usando-se o plural adequadamente. Exemplos:
\begin{itemize}
	\item \lq\lq Os resultados detalhados dos ensaios experimentais são 
	apresentados no Anexo IV, onde ...\rq\rq

	\item \lq\lq O Anexo I apresenta os resultados obtidos, onde pode-se 
	observar que ...\rq\rq

	\item \lq\lq Os Anexos I a IV apresentam os resultados obtidos ...\rq\rq

	\item \lq\lq Verificou-se uma forte dependência entre as variáveis citadas 
	(Anexo V), comprovando ...\rq\rq
\end{itemize}

