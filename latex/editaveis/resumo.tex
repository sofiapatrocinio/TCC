\begin{resumo}
    O aplicativo Multilind é um aplicativo desenvolvido por alunos durante as disciplinas de Métodos de Desenvolvimento de Software e Engenharia de Produto de Software do
    Curso de Engenharia de Software da Universidade de Brasília - Campus Gama, por nove membros, incluindo a autora desta monografia, sob a licença MIT. Além de contar com a parceria 
    da professora Altaci Corrêa Rubim, representante brasileira do Grupo de Trabalho Mundial da Década das Línguas Indígenas, e membros do GT do Brasil. Tem com o objetivo mapear as línguas indígenas presentes no Brasil, bem como fornecer informações a respeito das línguas, como família e tronco linguístico a que pertence, palavras 
    da língua e suas traduções para o português formal e indígena e imagens relativas. Após coletar avaliações de \textit{stakeholders}, foram identificados desafios relacionados à usabilidade e à experiência do usuário no aplicativo Multilind. Com base 
    nessas questões levantadas, o presente estudo tem como objetivo explorar como a aplicação de boas práticas de usabilidade e experiência de usuário podem contribuir para melhorias 
    em tais aspectos. Para isso, o aplicativo Multilind tem sido utilizado como estudo de caso, buscando compreender as percepções dos usuários em relação à versão atual do aplicativo. A partir dessas 
    percepções, foi elaborado um plano de melhorias adequado às demandas identificadas, desenvolvido a nível de protótipo de alta fidelidade. E, após validações junto aos usuários, serão 
    implementados aprimoramentos adicionais no aplicativo. Diante do exposto, espera-se que este trabalho forneça melhorias em relação à satisfação, engajamento e facilidade de uso do aplicativo Multilind.
    
 \vspace{\onelineskip}
    
 \noindent
 \textbf{Palavras-chave}: línguas indígenas. usabilidade. experiência do usuário. aplicativo \textit{mobile}.
\end{resumo}
