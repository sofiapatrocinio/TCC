\chapter[Introdução]{Introdução}

Neste capítulo, com base nos tópicos de interesse de pesquisa e desenvolvimento deste Trabalho de Conclusão de Curso, serão apresentados: uma breve
\hyperref[sec:Contextualização]{Contextualização}, conferindo informações sobre o domínio cognitivo Línguas Indígenas, e os conceitos Experiência de Usuário e
Usabilidade; a \hyperref[sec:Justificativa]{Justificativa} para realização desse trabalho, que consiste em refatorar um aplicativo já existente orientando-se por
Experiência de Usuário e Usabilidade; a \hyperref[sec:QuestaodePesquisa]{Questão de pesquisa/Desenvolvimento}, e os \hyperref[sec:Objetivos]{Objetivos},
\hyperref[sec:ObjetivoGeral]{Geral} e \hyperref[sec:ObjetivosEspecificos]{Específicos}. Por fim, tem-se a \hyperref[sec:OrganizacaodoTrabalho]{Organização do Trabalho}.

\section{Contextualização}
\label{sec:Contextualização}

Há uma grande dificuldade de mapeamento da quantidade de línguas indígenas existentes no Brasil. De acordo com os últimos dados censitários do IBGE, de 2010, o Brasil
possui 274 línguas indígenas faladas por indivíduos pertencentes a 305 etnias diferentes \cite{ibge}. Já de acordo com \cite{galucioetal2018} e \cite{dangelis2014}, esse
número estaria entre 160 e 170 línguas indígenas vivas.

De fato, não há certeza quanto ao número exato, o que se deve às dificuldades inerentes à definição técnica do que seja propriamente uma língua, agravadas pela falta de
informações sobre as línguas e seus falantes \cite{seki2000}.

Diante dessa realidade, o Grupo de Trabalho (GT) do Brasil \cite{gtbrasil2021}, constituído de uma equipe composta por representantes indígenas de cada região do Brasil,
lançou um plano de ação para a Década Internacional das Línguas Indígenas (IDL 2022-2032), decretada pela UNESCO, a fim de fortalecer políticas linguísticas, processos
socioculturais próprios, estratégias de educação, e circulação de conhecimentos dos povos indígenas de todo o mundo. O Plano de Ação para a Década Internacional das
Línguas Indígenas no Brasil aborda o lema “Nada para nós sem nós”, e estabelece a participação efetiva dos povos indígenas nos processos de tomada de decisão, consulta,
planejamento e implementação como princípios norteadores para a Década Internacional das Línguas Indígenas \cite{gtbrasil2021}.

Em parceria com a professora Altaci Corrêa Rubim\footnote{\url{https://amazoniareal.com.br/personagem/altaci-correa-rubim/}(último acesso: Novembro 2022)} - representante
brasileira do Grupo de Trabalho Mundial da Década das Línguas Indígenas - e membros do GT do Brasil, foi desenvolvido o
Multilind\footnote{\url{https://github.com/fga-eps-mds/2021.1-Multilind-Docs}}, aplicativo \textit{mobile} para mapeamento das línguas indígenas do Brasil. O aplicativo
foi desenvolvido durante as disciplinas de Métodos de Desenvolvimento de Software (MDS) e Engenharia de Produto de Software (EPS) do Curso de Engenharia de Software da
Universidade de Brasília - Campus Gama, por nove membros, sob a licença MIT. Entre suas funcionalidades, estão o mapeamento das línguas indígenas brasileiras, fornecendo
informações a respeito de cada língua como tronco linguístico a que pertence; local em que é falada; palavras da língua; bem como sua tradução para o português indígena e
o português formal, além da visualização de imagens associadas.

A aplicação foi desenvolvida utilizando \textit{React Native}, \textit{framework} escrito em \textit{JavaScript}, que é utilizado para escrever aplicações reais e de
renderização nativa, tanto para iOS quanto para \textit{Android} \cite{eiseman2017}.

O desenvolvimento do aplicativo foi guiado por testes de aceitação, cuja finalidade, segundo \citeonline{neto2015}, é a simulação de operações de rotina do sistema de
modo a verificar se seu comportamento está de acordo com o solicitado e foi realizado pela equipe do GT do Brasil.

Além da validação das histórias de usuários, os \textit{stakeholders} pontuaram sobre funcionalidades da aplicação que não foram projetadas de maneira efetiva aos
usuários, afetando a usabilidade do sistema.

Usabilidade é o termo técnico usado para descrever a qualidade de uso de uma interface \cite{bevan1995}. Trata-se de uma qualidade importante, pois interfaces com um
cuidado maior em termos de usabilidade aumentam a produtividade dos usuários, diminuem a ocorrência de erros e, por consequência, contribuem para a satisfação dos
usuários \cite{winckler2022}.

A respeito da usabilidade, não se pode deixar de citar a importância das heurísticas de Nielsen, diretrizes elaboradas para garantir que o usuário tenha uma boa
experiência de uso. São elas: Visibilidade do status do sistema; compatibilidade do sistema com o mundo real; controle do usuário e liberdade; consistência e padrões;
ajuda os usuários a reconhecer, diagnosticar e recuperação de erros; prevenção de erros; reconhecimento em vez de memorização; flexibilidade e eficiência de uso; estética
e design minimalista e ajuda e documentação \cite{nielsen1994}.

Ademais, é importante entender que a usabilidade pode variar em função de alguns critérios e está relacionada ao tipo de aplicação em questão, perfil dos usuários,
contextos de utilização, etc., que são variáveis \cite{winckler2022}. Considerando o perfil dos usuários, focado principalmente em pessoas indígenas, é importante também
compreender a importância de projetar interfaces acessíveis. De acordo com \citeonline{takashi2000}:

\begin{quote}
    Outro fator de dificuldade para o usuário inexperiente é o desenho das telas de apresentação e a estruturação das páginas, muitas vezes pressupondo uma certa
    familiaridade com ambientes computacionais mais sofisticados.

\end{quote}

\section{Justificativa}
\label{sec:Justificativa}

O acesso às Tecnologias de Informação e Comunicação pelos povos indígenas é um marco do período recente e a inclusão social e digital de minorías étnicas, especificamente
indígenas, por muito tempo estiveram à margem do acesso/uso da informação \cite{pinto2010}. Além disso, com o passar do tempo, muitas culturas indígenas estão esquecendo
seus conhecimentos, e sendo absorvidas pela cultura dominante nacional, resultando em um fenômeno social conhecido como hibridismo ou miscigenação, que pode ir contra seu
conhecimento autóctone \cite{pinto2010}.

Por isso, é de extrema importância a documentação, reprodução e divulgação das informações a respeito de línguas indígenas que, de acordo com \cite{moore2008}, pelo menos
25\% delas estariam ameaçadas de extinção em um futuro próximo por conta do baixo número de falantes e falta de transmissão. O desaparecimento dessas línguas seria uma
grande perda para as comunidades nativas, dado que são os meios de transmissão da cultura e pensamento tradicionais e parte importante da identidade étnica.
\cite{moore2008}.

Visando à preservação e à revitalização das diversas línguas indígenas existentes no Brasil, este projeto de pesquisa visa à aplicação de boas práticas de usabilidade,
atendendo as necessidades do usuário e garantindo uma boa experiência de uso.


\section{Questão de Pesquisa}
\label{sec:QuestaodePesquisa}

A partir dos problemas apresentados, o presente trabalho tem a pretensão de responder à seguinte questão de pesquisa:

Como proporcionar uma aplicação com boas práticas de usabilidade, a fim de auxiliar na revitalização e preservação das línguas indígenas, voltadas às pessoas indígenas?

\section{Objetivos}
\label{sec:Objetivos}

\subsection{Objetivo Geral}
\label{sec:ObjetivoGeral}

O objetivo principal deste trabalho é realizar melhorias centradas na experiência de usuário no aplicativo Multilind.

\subsection{Objetivos Específicos}
\label{sec:ObjetivosEspecificos}

\begin{description}
    \item A partir do Objetivo Geral, pôde-se definir os Objetivos Específicos, sendo:
          \begin{itemize}
              \item Identificar as tecnologias de design, prototipação e teste que permitam atender às finalidades do trabalho;

              \item Aplicar conceitos de boas práticas de usabilidade no produto de software;

              \item Planejar a avaliação empírica sob a perspectiva da experiência de usuário;

              \item Realizar testes de usabilidade no processo avaliativo, procurando coletar e manter os rastros dos resultados obtidos e

              \item Documentar os resultados do estudo exploratório, bem como prover uma primeira análise dos resultados obtidos.
          \end{itemize}
\end{description}

\section{Organização do Trabalho}
\label{sec:OrganizacaodoTrabalho}

\begin{description}
    \item Esta monografia está disposta em capítulos, conforme apresentado a seguir:
          \begin{itemize}
              \item Capítulo 2 - Referencial Teórico: apresenta os fundamentos teóricos do trabalho e conceitos relacionados à experiência de usuário e usabilidade.

              \item Capítulo 3 - Suporte Tecnológico: define as tecnologias e ferramentas utilizadas no trabalho, no que diz respeito a prototipação, desenvolvimento,
                    avaliação e validação do software e de pesquisa e escrita;

              \item Capítulo 4 - Metodologia: especifica os aspectos metodológicos sobre o levantamento bibliográfico, o desenvolvimento do software e a análise de
              resultados, além de apresentar o cronograma de atividades;

              \item Capítulo 5 - Proposta: detalha a proposta deste trabalho e descreve a estrutura e a aplicabilidade da solução apresentada e

              \item Capítulo 6 - Considerações Finais: apresenta os resultados alcançados no TCC 01.
          \end{itemize}
\end{description}
