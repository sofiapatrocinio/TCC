\setcounter{chapter}{1}
\chapter*[Introdução]{Introdução}
\addcontentsline{toc}{chapter}{Introdução}
\section{Introdução}

Nesse capítulo, com base nos tópicos de interesse de pesquisa e desenvolvimento desse Trabalho de Conclusão de Curso, será apresentada um breve contexto em que o trabalho
proposto se insere. Além de apresentar seu tema foco, justificativas e objetivos e por fim, a organização do trabalho.

\subsection{Contextualização}

Há uma grande dificuldade de mapeamento da quantidade de línguas indígenas existentes no Brasil. De acordo com os últimos dados censitários do IBGE, de 2010, o Brasil
possui 274 línguas indígenas faladas por indivíduos pertencentes a 305 etnias diferentes. \cite{ibge}. Já de acordo com \cite{galucioetal2018} e
\cite{dangelis2014}, esse número estaria entre 160 e 170 línguas indígenas vivas.

De fato, não há certeza quanto ao número exato, o que se deve às dificuldades inerentes à definição técnica do que seja propriamente uma língua, agravadas pela falta de
informações sobre as línguas e seus falantes. \cite{seki2000}.

Diante dessa realidade, o Grupo de Trabalho (GT) do Brasil, constituído de uma equipe composta por representantes indígenas de cada região do Brasil, lançou um plano de
ação para a Década Internacional das Línguas Indígenas (IDL 2022-2032), decretada pela UNESCO a fim de fortalecer políticas linguísticas, processos socioculturais
próprios e estratégias de educação e circulação de conhecimentos dos povos indígenas de todo o mundo. O Plano de Ação para a Década Internacional das Línguas Indígenas no
Brasil aborda o lema “Nada para nós sem nós” e estabelece a participação efetiva dos povos indígenas nos processos de tomada de decisão, consulta, planejamento e
implementação como princípios norteadores para a Década Internacional das Línguas Indígenas. \cite{gtbrasil2021}.

Em parceria com a professora Altaci Corrêa Rubim, representante brasileira do Grupo de Trabalho Mundial da Década das Línguas Indígenas e membros do GT do Brasil, foi
desenvolvido o Multilind, aplicativo \textit{mobile} para mapeamento das línguas indígenas do Brasil. O aplicativo foi desenvolvido durante as disciplinas de Métodos de
Desenvolvimento de Software (MDS) e Engenharia de Produto de Software (EPS) do Curso de Engenharia de Software da Universidade de Brasília - Campus Gama, por 9 membros,
sob a licença MIT. Entre suas funcionalidades estão o mapeamento das línguas indígenas brasileiras, fornecendo informações a respeito de cada língua como tronco
linguístico a que pertence; local em que é falada, palavras da língua, bem como sua tradução para o português indígena e português formal, além da visualização de imagens
associadas.
