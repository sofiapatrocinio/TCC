\chapter[Introdução]{Introdução}

Neste capítulo, com base nos tópicos de interesse de pesquisa e desenvolvimento deste Trabalho de Conclusão de Curso, serão apresentados: uma breve
\hyperref[sec:Contextualização]{Contextualização}, conferindo informações sobre o domínio cognitivo Línguas Indígenas, e os conceitos Experiência de Usuário e
Usabilidade; a \hyperref[sec:Justificativa]{Justificativa} para realização desse trabalho, que consiste em refatorar um aplicativo já existente orientando-se por
Experiência de Usuário e Usabilidade; a \hyperref[sec:QuestaodePesquisa]{Questão de Pesquisa}, e os \hyperref[sec:Objetivos]{Objetivos},
\hyperref[sec:ObjetivoGeral]{Geral} e \hyperref[sec:ObjetivosEspecificos]{Específicos}. Por fim, tem-se a \hyperref[sec:OrganizacaodaMonografia]{Organização da Monografia}.

\section{Contextualização}
\label{sec:Contextualização}

Há uma grande dificuldade de mapeamento da quantidade de línguas indígenas existentes no Brasil. De acordo com os últimos dados censitários do IBGE, de 2010, o Brasil
possui 274 línguas indígenas faladas por indivíduos pertencentes a 305 etnias diferentes \cite{ibge}. Já de acordo com \citeonline{galucioetal2018} e \cite{dangelis2014}, esse
número estaria entre 160 e 170 línguas indígenas vivas.

De fato, não há certeza quanto ao número exato, o que se deve às dificuldades inerentes à definição técnica do que seja propriamente uma língua, agravadas pela falta de
informações sobre as línguas e seus falantes \cite{seki2000}.

Diante dessa realidade, o Grupo de Trabalho (GT) do Brasil \cite{gtbrasil2021}, constituído de uma equipe composta por representantes indígenas de cada região do Brasil,
lançou um plano de ação para a Década Internacional das Línguas Indígenas (IDL 2022-2032), decretada pela UNESCO, a fim de fortalecer políticas linguísticas, processos
socioculturais próprios, estratégias de educação, e circulação de conhecimentos dos povos indígenas de todo o mundo. O Plano de Ação para a Década Internacional das
Línguas Indígenas no Brasil aborda o lema “Nada para nós sem nós”, e estabelece a participação efetiva dos povos indígenas nos processos de tomada de decisão, consulta,
planejamento e implementação como princípios norteadores para a Década Internacional das Línguas Indígenas \cite{gtbrasil2021}.

Em parceria com a professora Altaci Corrêa Rubim\footnote{\url{https://amazoniareal.com.br/personagem/altaci-correa-rubim/}(último acesso: Abril 2023)} - representante
brasileira do Grupo de Trabalho Mundial da Década das Línguas Indígenas - e membros do GT do Brasil, foi desenvolvido o
Multilind\footnote{\url{https://github.com/fga-eps-mds/2021.1-Multilind-Docs}(último acesso: Abril 2023)}, aplicativo \textit{mobile} para mapeamento das línguas
indígenas do Brasil. O aplicativo foi desenvolvido durante as disciplinas de Métodos de Desenvolvimento de Software (MDS) e Engenharia de Produto de Software (EPS) do
Curso de Engenharia de Software da Universidade de Brasília - Campus Gama, por nove membros, incluindo a autora dessa monografia, sob a licença MIT. Entre suas funcionalidades, estão o mapeamento das línguas
indígenas brasileiras, fornecendo informações a respeito de cada língua como tronco linguístico a que pertence; local em que é falada; palavras da língua; bem como sua
tradução para o português indígena e o português formal, além da visualização de imagens associadas.

A aplicação foi desenvolvida utilizando \textit{React Native}, \textit{framework} escrito em \textit{JavaScript}, que é utilizado para escrever aplicações reais e de
renderização nativa, tanto para iOS quanto para \textit{Android} \cite{eiseman2017}.

O desenvolvimento do aplicativo foi guiado por testes de aceitação, cuja finalidade, segundo \citeonline{neto2015}, é a simulação de operações de rotina do sistema de
modo a verificar se seu comportamento está de acordo com o solicitado e realizado pela equipe do GT do Brasil.

Além da validação das histórias de usuários, os \textit{stakeholders} pontuaram sobre funcionalidades da aplicação que não foram projetadas de maneira efetiva aos
usuários, afetando a usabilidade, e por consequência, a experiência do usuário ao utilizar a aplicação.

Experiência de usuário pode ser definida como um conjunto de aspectos da interação do usuário com um produto, aplicação ou sistema. Considera fatores como pensamentos,
sentimentos e percepções que resultam dessa interação \cite{tulis2013}.

Usabilidade é o termo técnico usado para descrever a qualidade de uso de uma interface \cite{bevan1995}. Trata-se de uma qualidade importante, pois interfaces com um
cuidado maior em termos de usabilidade aumentam a produtividade dos usuários, diminuem a ocorrência de erros e, por consequência, contribuem para a satisfação dos
usuários \cite{winckler2022}.

A respeito da usabilidade, não se pode deixar de citar a importância das heurísticas de Nielsen \cite{nielsen1994}, sendo essas diretrizes elaboradas para garantir que o
usuário tenha uma boa experiência de uso. As heurísticas\footnote{10 Usability Heuristics for User Interface Design. Portal NN/g Nielsen Norman Group, 2020. Disponível
em: \url{https://www.nngroup.com/articles/ten-usability-heuristics/} (último acesso: Abril 2023)} são: \#1 visibilidade do status do sistema; \#2 compatibilidade do
sistema com o mundo real; \#3 controle do usuário e liberdade; \#4 consistência e padrões; \#5 prevenção de erros; \#6 reconhecimento em vez de memorização; \#7
flexibilidade e eficiência de uso; \#8 estética e \textit{design} minimalista; \#9 ajuda aos usuários reconhecerem, diagnosticarem e se recuperarem de erros, e \#10 ajuda e 
documentação.

Ademais, é importante entender que a usabilidade pode variar em função de alguns critérios, como eficácia, eficiência e satisfação, e está relacionada ao tipo de
aplicação em questão; perfil dos usuários; contextos de utilização, dentre outros, os quais são variáveis \cite{winckler2022}. Nesse sentido, e dado o público alvo desse
trabalho, ou seja, pessoas indígenas, é importante também compreender a importância de projetar interfaces acessíveis. Por fim, mas não menos relevante, conforme colocado 
por \citeonline{takashi2000}, outro fator de dificuldade para o usuário inexperiente é o desenho das telas de apresentação e a estruturação das páginas, muitas vezes 
pressupondo uma certa familiaridade com ambientes computacionais mais sofisticados.

\section{Justificativa}
\label{sec:Justificativa}

O acesso às Tecnologias de Informação e Comunicação pelos povos indígenas é um marco do período recente. A inclusão social e digital de minorías étnicas, especificamente
indígenas, por muito tempo estiveram à margem do acesso/uso da informação \cite{pinto2010}. Além disso, com o passar do tempo, muitas culturas indígenas estão esquecendo
seus conhecimentos, e sendo absorvidas pela cultura dominante nacional, resultando em um fenômeno social conhecido como hibridismo ou miscigenação, que pode ir contra seu
conhecimento autóctone \cite{pinto2010}.

Documentação, reprodução e divulgação de informações sobre línguas indígenas é de suma importância, conforme observado por \citeonline{moore2008},
pelo menos 25\% delas estariam ameaçadas de extinção em um futuro próximo por conta do baixo número de falantes e pela falta de transmissão. O desaparecimento dessas línguas
seria uma grande perda para as comunidades nativas, dado que são os meios de transmissão da cultura e dos pensamentos tradicionais e parte importante da identidade étnica
\cite{moore2008}.

Diante do exposto, manter o aplicativo Multilind em evolução, aprimorando-o constantemente em aspectos que ele demonstra fragilidades, representa uma forma de
contribuir para a reprodução e a divulgação de línguas indígenas, adicionalmente, preservando e revitalizando-as. Este trabalho visa, portanto,
refatorar o aplicativo Multilind, aplicando boas práticas de usabilidade, e procurando impactar positivamente a experiência de usuário no aplicativo.

\section{Questão de Pesquisa}
\label{sec:QuestaodePesquisa}

O presente trabalho pretende conferir insumos visando responder a seguinte questão de pesquisa:

É  possível melhorar a experiência de usuário no aplicativo Multilind aplicando boas práticas de usabilidade? 

A Questão de Pesquisa parte da premissa de que usabilidade impacta positivamente na experiência de usuário. Ressalta-se ainda que o principal perfil de usuários do aplicativo 
representa pessoas indígenas, sendo necessárias validações considerando uma amostra desse público alvo e foco em experiência de usuário.

\section{Objetivos}
\label{sec:Objetivos}

O principal objetivo deste trabalho, visto como o Objetivo Geral, é refatorar o aplicativo Multilind, procurando melhorá-lo em termos de experiência de usuário, e
orientando-se por boas práticas de usabilidade. 

\begin{description}
    \item A fim de atingir o Objetivo Geral, pretende-se cumprir com os seguintes Objetivos Específicos:
          \begin{itemize}
              \item Especificação clara do público alvo, sendo esse centrado em pessoas indígenas, evidenciando esse perfil de usuários com uso de personas;

              \item Levantamento de referenciais teóricos, no que tange os conceitos de usabilidade e experiência de usuário;

              \item Detalhamento de tecnologias e outros recursos técnicos que sejam adequados para a refatoração do aplicativo Multilind;

              \item Uso de prototipação, visando validações pontuais junto aos interessados, e facilitando a refatoração do aplicativo;

              \item Condução da análise de resultados, e
              
              \item Documentação do trabalho como um todo, orientando-se por metodologias de cunho investigativo (Pesquisa Bibliográfica), de desenvolvimento 
              (uma adaptação orientada a métodos ágeis), e de análise de resultados (Pesquisa-Ação, com uso do Questionário AttrakDiff \cite{natashatayana2015}
              \cite{hassenzahl2003}).
          \end{itemize}
\end{description}

\section{Organização da Monografia}
\label{sec:OrganizacaodaMonografia}

\begin{description}
    \item Esta monografia está disposta em capítulos, conforme apresentado a seguir:
          \begin{itemize}
              \item Capítulo 2 - Referencial Teórico: apresenta os fundamentos teóricos do trabalho e conceitos relacionados à experiência de usuário e usabilidade.

              \item Capítulo 3 - Suporte Tecnológico: define as tecnologias e ferramentas utilizadas no trabalho, no que diz respeito à refatoração (ou seja, parte
              atrelada ao desenvolvimento), à avaliação dos resultados (ex. prototipacão e testes), e às pesquisa e escrita;

              \item Capítulo 4 - Metodologia: esclarece sobre a classificação da pesquisa orientando-se pela literatura especializada; especifica os aspectos
              metodológicos sobre a pesquisa bibliográfica, o desenvolvimento do software e a análise de resultados, além de apresentar o cronograma de atividades;

              \item Capítulo 5 - Proposta: apresenta a proposta deste trabalho, procurando detalhá-la em termos de origem da ideia; público alvo; aplicativo Multilind;
              necessidade de melhorias identificadas no aplicativo; \textit{baseline} de boas práticas de usabilidade em atendimento às demandas de melhorias, dentre
              outros aspectos, e

              \item Capítulo 6 - Considerações Finais: contempla os resultados alcançados na primeira etapa do TCC, retomando os Objetivos Específicos, e evidenciando quais já foram
              cumpridos; quais estão em andamento, e quais serão cumpridos no escopo da segunda etapa do TCC.
          \end{itemize}
\end{description}