\begin{resumo}[Abstract]
 \begin{otherlanguage*}{english}
   
  The Multilind application is an app to maps the indigenous languages present in Brazil and provides information about these languages, such as the language family and 
  language trunk, language words, their translations in both formal Portuguese and indigenous languages, and related images. It was developed by students during the Software Development Methods and Software Product Engineering courses of the Software Engineering program at the University of Brasília - Campus Gama, 
  by nine members, including the author of this monography. It also has the partnership of Professor Altaci Corrêa Rubim, the Brazilian representative of the World Working Group of the Decade 
  of Indigenous Languages, and members of the Brazilian Working Group. After gathering feedback from stakeholders, challenges related to usability and user experience 
  in the Multilind application have been identified. Based on these raised concerns, this study explored how the application of usability and user experience best practices can contribute to improvements in these aspects. 
  To achieve this, the Multilind application was used as a case study to understand user perceptions of the current version of the app. Using these perceptions, an improvement plan to identify demands has been developed 
  in the form of a high-fidelity prototype. Following user validations, additional enhancements was implemented in the application. Given the above, and the results achieved, it is believed that this work provided improvements 
  in terms of user satisfaction, engagement, and ease of use of the Multilind application.

   \vspace{\onelineskip}
 
   \noindent 
   \textbf{Key-words}: Indigenous Languages. Usability. User Experience. Mobile Application. Improvements.
 \end{otherlanguage*}
\end{resumo}
