\begin{resumo}[Abstract]
 \begin{otherlanguage*}{english}
   
  The Multilind application is an app developed by students during the Software Development Methods and Software Product Engineering courses of the Software Engineering program at the University of Brasília - Campus Gama, 
  by nine members, including the author of this monograph, under the MIT license. In addition, it has the partnership of Professor Altaci Corrêa Rubim, the Brazilian representative of the World Working Group for the Decade 
  of Indigenous Languages, and members of the Brazilian Working Group. Its objective is to map the indigenous languages present in Brazil, as well as provide information about the languages, such as their language family and 
  linguistic branch, language words, and their translations into formal Portuguese and indigenous Portuguese, along with relevant images. After collecting stakeholder feedback, challenges related to usability and user experience 
  in the Multilind application were identified. Based on these raised issues, this study aims to explore how the application of good usability and user experience practices can contribute to improvements in these aspects. 
  For this purpose, the Multilind application has been used as a case study, seeking to understand users' perceptions of the current version of the app. Based on these perceptions, a suitable improvement plan was developed 
  to address the identified demands, which was then implemented as a high-fidelity prototype. After validation with users, additional enhancements will be implemented in the application. In light of the above, it is expected 
  that this work will provide improvements in terms of user satisfaction, engagement, and ease of use of the Multilind application.


   \vspace{\onelineskip}
 
   \noindent 
   \textbf{Key-words}: indigenous languages, usability, user experience, mobile application.
 \end{otherlanguage*}
\end{resumo}
