\chapter[Metodologia]{Metodologia}
O objetivo deste capítulo é fornecer uma visão geral das metodologias utilizadas ao longo do trabalho nas etapas de escrita, exploração teórica e desenvolvimento prático. 
A estrutura do capítulo se inicia pela Metodologia da Pesquisa Científica, considerando Abordagem, Natureza, Objetivos e Procedimentos. Em seguida, tem-se o Fluxo de Atividades que guia o desenvolvimento  
da monografia, Metodologia de Desenvolvimento e de Análise de Resultados. Por fim, é definido o Cronograma de Atividades e tem-se o Resumo do Capítulo, que possui uma visão completa das etapas 
metodológicas abordadas.

\section{Classificação da Pesquisa}
\label{sec:Classificação da Pesquisa}
De acordo com Gil (1999), pode-se definir pesquisa como o procedimento racional e sistemático que tem por objetivo proporcionar respostas aos problemas que são propostos. É desenvolvida através da colaboração 
dos conhecimentos existentes e a aplicação cuidadosa de métodos, técnicas e outros procedimentos científicos.

Com relação as escolhas metodológicas, podem ser utilizadas classificações quanto ao objetivo da pesquisa, quanto a natureza da pesquisa e quanto a escolha do objeto de estudo. O objetivo desta seção é 
categorizar a investigação realizada neste estudo com base nesses critérios.

\subsection{Objetivos}
\label{sec:Objetivos2}
Segundo Malhotra (2001), as pesquisas podem ser classificadas, em termos amplos, como exploratórias ou conclusivas. Sendo a pesquisa exploratória caracterizada por Gil (1999) pelo seu principal objetivo de 
desenvolver, esclarecer e modificar conceitos e ideias, com a finalidade de formular problemas mais precisos ou hipóteses pesquisáveis para estudos futuros.

Visando explorar a usabilidade e experiência de usuário de uma aplicação específica, com público alvo definido, pode-se classificar como Pesquisa Exploratória, com objetivo de proporcionar maior familiaridade 
com o assunto pesquisado.

Este tipo de pesquisa tem como objetivo proporcionar maior familiaridade com o problema, com vistas a torná-lo mais explícito ou a construir hipóteses. A grande maioria dessas pesquisas envolve: (a) levantamento 
bibliográfico; (b) entrevistas com pessoas que tiveram experiências práticas com o problema pesquisado; e (c) análise de exemplos que estimulem a compreensão (GIL, 2007).

\subsection{Natureza}
\label{sec:Natureza}
Gerhardt e Silveira (2009) propõem uma classificação do ponto de vista da natureza de pesquisa em dois grupos distintos:

\begin{itemize}
	\item Pesquisa Básica - tem como objetivo gerar conhecimentos novos que sejam úteis para o avanço da ciência, sem uma aplicação prática prevista.
	\item Pesquisa Aplicada - tem como finalidade gerar conhecimentos direcionados à aplicação prática, com foco na resolução de problemas específicos.
\end{itemize}

Quanto à natureza, a pesquisa deste trabalho pode ser classificada como Aplicada, pois tem aplicação prática de melhoria em uma aplicação específica.