\chapter[Metodologia]{Metodologia}
\label{chap:Metodologia}

O objetivo deste capítulo é fornecer uma visão geral das metodologias utilizadas ao longo do trabalho nas etapas de escrita, exploração teórica e desenvolvimento prático. 
A estrutura do capítulo se inicia pela Metodologia da Pesquisa Científica, considerando Abordagem, Natureza, Objetivos e Procedimentos. Em seguida, tem-se o Fluxo de Atividades que guia o desenvolvimento  
da monografia, Metodologia de Desenvolvimento e de Análise de Resultados. Por fim, é definido o Cronograma de Atividades e tem-se o Resumo do Capítulo, que possui uma visão completa das etapas 
metodológicas abordadas.

\section{Classificação da Pesquisa}
\label{sec:Classificação da Pesquisa}
De acordo com Gil (1999), pode-se definir pesquisa como o procedimento racional e sistemático que tem por objetivo proporcionar respostas aos problemas que são propostos. É desenvolvida através da colaboração 
dos conhecimentos existentes e a aplicação cuidadosa de métodos, técnicas e outros procedimentos científicos.

Com relação as escolhas metodológicas, podem ser utilizadas classificações quanto ao objetivo da pesquisa, quanto a natureza da pesquisa e quanto a abordagem do problema. O objetivo desta seção é 
categorizar a investigação realizada neste estudo com base nesses critérios.

\subsection{Objetivos}
\label{sec:Objetivos2}
Segundo Malhotra (2001), as pesquisas podem ser classificadas, em termos amplos, como exploratórias ou conclusivas. Sendo a pesquisa exploratória caracterizada por Gil (1999) pelo seu principal objetivo de 
desenvolver, esclarecer e modificar conceitos e ideias, com a finalidade de formular problemas mais precisos ou hipóteses pesquisáveis para estudos futuros.

Visando explorar a usabilidade e experiência de usuário de uma aplicação específica, com público alvo definido, pode-se classificar como Pesquisa Exploratória, com objetivo de proporcionar maior familiaridade 
com o assunto pesquisado.

Este tipo de pesquisa tem como objetivo proporcionar maior familiaridade com o problema, com vistas a torná-lo mais explícito ou a construir hipóteses. A grande maioria dessas pesquisas envolve: (a) levantamento 
bibliográfico; (b) 	entrevistas com pessoas que tiveram experiências práticas com o problema pesquisado; e (c) análise de exemplos que estimulem a compreensão (GIL, 2007).

\subsection{Natureza}
\label{sec:Natureza}
Gerhardt e Silveira (2009) propõem uma classificação do ponto de vista da natureza de pesquisa em dois grupos distintos:

\begin{itemize}
	\item Pesquisa Básica - tem como objetivo gerar conhecimentos novos que sejam úteis para o avanço da ciência, sem uma aplicação prática prevista.
	\item Pesquisa Aplicada - tem como finalidade gerar conhecimentos direcionados à aplicação prática, com foco na resolução de problemas específicos.
\end{itemize}

Quanto à natureza, a pesquisa deste trabalho pode ser classificada como Aplicada, pois tem aplicação prática de melhoria em uma aplicação específica.

\subsection{Abordagem}
\label{sec:Abordagem}
Do ponto de vista da abordagem do problema, pode-se classificar, de acordo com Gil (1999), como pesquisa quantitativa e qualitativa. A pesquisa quantitativa traduz opiniões e informações em números para classificá-las e analisá-las. 
Já a pesquisa qualitativa considera as subjetividades entre o sujeito e o mundo real e não é traduzida em números, e tem como principais focos o processo e seu significado.

A abordagem dessa pesquisa pode ser classificada tanto como qualitativa quanto quantitativa. No aspecto qualitativo, são considerados fatores como pensamentos, sentimentos e percepções, relacionados à experiência 
do usuário. Por outro lado, no aspecto quantitativo, a análise da usabilidade envolve testes com métricas mensuráveis, como a quantidade de cliques errados em uma determinada tela.

\subsection{Procedimentos}
\label{sec:Procedimentos}
Do ponto de vista dos procedimentos técnicos (GIL, 1991), a pesquisa pode ser classificada da seguinte forma:

\begin{itemize}
	\item Pesquisa Bibliográfica: baseada em material já publicado, como livros, artigos de periódicos e conteúdo disponibilizado na Internet.
	\item Pesquisa Documental: utiliza materiais que não passaram por análise, como documentos, arquivos e registros.
	\item Pesquisa Experimental: determina um objeto de estudo, seleciona variáveis que podem influenciá-lo, estabelece formas de controle e observação dos efeitos produzidos pelas variáveis no objeto.
	\item Levantamento: envolve a coleta direta de informações por meio de questionários ou entrevistas, com o objetivo de conhecer o comportamento das pessoas.
	\item Estudo de caso: realiza uma análise detalhada e aprofundada de um ou poucos objetos, permitindo um amplo conhecimento dos mesmos.
	\item Pesquisa Expost-Facto: ocorre após os eventos ocorrerem, realizando um "experimento" posterior aos fatos.
	\item Pesquisa-Ação: concebida e realizada em associação direta com uma ação ou resolução de um problema coletivo, com a participação colaborativa dos pesquisadores e dos participantes envolvidos na situação ou problema.
	\item Pesquisa Participante: desenvolve-se por meio da interação entre pesquisadores e membros das situações investigadas.
\end{itemize}

Essa pesquisa pode ser classificada em três categorias: pesquisa bibliográfica, pesquisa-ação e estudo de caso. A pesquisa bibliográfica consiste em realizar levantamentos bibliográficos nas áreas de usabilidade e experiência do usuário, 
utilizando esses conhecimentos como base para a realização do estudo; a pesquisa-ação é um termo amplo que descreve um ciclo de melhoria prática que percorre os campos da prática e da investigação, geralmente envolvendo etapas de planejamento, 
implementação, descrição e avaliação (TRIPP, 2005), e o estudo de caso envolve a análise aprofundada de uma entidade específica, neste caso, o aplicativo Multilind, com o objetivo de compreender o funcionamento do contexto a partir da perspectiva 
dos usuários (GERHARDT; SILVEIRA, 2009).

\section{Fluxo de Atividades}
\label{sec:Fluxo de Atividades}
A condução dessa monografia requer a definição de um fluxo de atividades estruturado em diversas etapas fundamentais. Essas etapas são representadas e organizadas no modelo ilustrado na Figura \ref{fig04}, seguindo a notação BPMN (Business Process Model and Notation). 
No contexto do primeiro estágio da elaboração do TCC, temos as seguintes atividades e subprocessos:

\begin{itemize}
	\item Definir Tema: Atividade destinada à escolha do tema de desenvolvimento do trabalho. Com o auxílio da orientadora e visando uma das áreas de interesse da autora, foi definido o tema: Aplicativo Multilind: Melhoria Orientada à
	Experiência de Usuário e à Usabilidade. \textit{Status}: \textbf{Concluída}.
	\item Contextualizar Problema: Atividade que consiste em desenvolver uma visão clara em relação ao contexto do aplicativo e o problema que visa resolver. \textit{Status}: \textbf{Concluída} - \hyperref[chap:Introducao]{Capítulo 1}.
	\item Levantamento Bibliográfico: Subprocesso responsável pela consulta a bases científicas para investigar a literatura especializada sobre Usabilidade, Experiência de Usuário e seus desdobramentos. \textit{Status}: \textbf{Concluída} - \hyperref[chap:Referencial]{Capítulo 2}.
	\item Definir Suporte Tecnológico: Procedimento que visa a definição de ferramentas e tecnologias que apoiam a realização dessa monografia. \textit{Status}: \textbf{Concluída} - \hyperref[chap:ReferencialTech]{Capítulo 3}.
	\item Especificar Metodologia: Subprocesso que trata de aspectos metodológicos da pesquisa e desenvolvimento do trabalho. \textit{Status}: \textbf{Concluída} - \hyperref[chap:Metodologia]{Presente Capítulo}.
	\item Coletar Informações à respeito de Usabilidade e UX do aplicativo: Atividade com o objetivo de levantar métricas para a realização de melhorias na Usabilidade e Experiência de Usuário. \textit{Status}: \textbf{Pendente}.
	\item Desenvolver Protótipo baseado em Melhorias: Criação de protótipo de alta fidelidade levando em consideração as informações levantadas e na literatura. \textit{Status}: \textbf{Pendente}.
	\item Determinar Proposta do Trabalho: Detalhamento da proposta, considerando documentação dos dados coletados antes e depois das melhorias na aplicação, prova de conceito e afins. \textit{Status}: \textbf{Pendente}.
	\item Revisar TCC 1: Atividade que abrange todos os aspectos do TCC, incluindo sua escrita e os artefatos produzidos. Essa atividade é realizada por meio de iterações, tanto da autora quanto da orientadora. \textit{Status}: \textbf{Em desenvolvimento}.
	\item Apresentar TCC 1: Apresentação da primeira parte da monografia para a banca avaliadora. \textit{Status}: \textbf{Pendente}.
\end{itemize}

Em relação as atividades e subprocessos do segundo estágio da elaboração do TCC, tem-se:

\begin{itemize}
	\item Corrigir Apontamentos da Banca: Atividade com objetivo de refinar a monografia baseada nas considerações da banca avaliadora. \textit{Status}: \textbf{A Realizar}.
	\item Desenvolvimento da Aplicação: Subprocesso voltado ao desenvolvimento das melhorias de Usabilidade e Experiência do Usuário. O detalhamento desse subprocesso está disponível na Seção X. \textit{Status}: \textbf{A Realizar}.
	\item Análise de Resultados: Subprocesso que consiste na coleta de métricas e realização de melhorias com base nelas. Seu detalhamento está explicitado na Seção X. \textit{Status}: \textbf{A Realizar}.
	\item Revisar TCC 2: Atividade voltada ao aperfeiçoamento da escrita e desenvolvimento da monografia de forma iterativa. \textit{Status}: \textbf{A Realizar}.
	\item Apresentar TCC 2: Apresentação da monografia em sua totalidade para a banca avaliadora. \textit{Status}: \textbf{A Realizar}.
\end{itemize}

\begin{figure}[h!]
	\centering
	\resizebox{1\textwidth}{!}{\includegraphics{figuras/tcc1.eps}}
	\caption{Fluxo de Atividades TCC 1}
	\label{fig04}
\end{figure}

\section{Levantamento Bibliográfico}
\label{sec:Levantamento Bibliografico}
De acordo com Gil (2002), o levantamento bibliográfico preliminar é fundamental na elaboração da pesquisa, pois proporciona familiaridade com a área de estudo. No presente trabalho, foi realizado com base em artigos, livros e outras monografias a fim de solidificar sua base teórica.

Para garantir a qualidade e atualidade das fontes utilizadas, as plataformas de base científica Periódicos Capes e Science Direct, que se destacam pela revisão em pares de seus artigos e Google Scholar, que garante uma abordagem ampla que inclui livros e outras publicações acadêmicas. 

\subsection{\textit{Strings} de Busca}
\label{sec:Strings de Busca}
Após a definição do tema, foram estabelecidas as \textit{strings} de busca para realizar pesquisas mais direcionadas nas principais bases de dados. A relação de busca por base de dados e seus resultados pode ser vista na Tabela X.

\begin{table}[h]
	\centering
	\begin{tabular}{|l|l|l|}
	\hline
	\textit{String}               & Base de Dados            & Resultados \\ \hline
	'\textit{Software Usability}' & Periódico Capes          & 19.578     \\ \hline
	'\textit{User Experience}'    & Periódico Capes          & 159.636    \\ \hline
	'\textit{Usability Test}'     & Periódico Capes          & 15.082     \\ \hline
	'\textit{User Experience}'    & ScienceDirect (Elsevier) & 597.433    \\ \hline
	'\textit{Usability}'          & ScienceDirect (Elsevier) & 112        \\ \hline
	'\textit{Software Usability}' & Google Scholar           & 13.400     \\ \hline
	'\textit{User Experience}'    & Google Scholar           & 854.000    \\ \hline
	'\textit{Usability Test}'     & Google Scholar           & 39.300     \\ \hline
	\end{tabular}
\end{table}

\subsection{Critérios de Seleção}
\label{sec:Critérios de Selecao}
Durante a análise dos artigos, livros e periódicos relacionados aos temas descritos na Seção X, foram estabelecidos critérios para refinar os materiais. Os critérios de seleção utilizados para esse refinamento foram:

\begin{itemize}
	\item Ter sido escrito em português, inglês ou alemão;
	\item Tratar métricas relacionado à Experiência de Usuário e Usabilidade;
\end{itemize}

Com base nisso, alguns dos principais artigos selecionados foram:

\begin{itemize}
	\item \textit{Usability is quality of use} \cite{bevan1995}
	\item \textit{User experience - a research agenda} \cite{hassenzahl2006}
	\item \textit{Attrakdiff: Ein fragebogen zur messung wahrgenommener hedonischer und pragmatischer qualität} \cite{hassenzahl2003}
	\item \textit{Usability Engineering} \cite{nielsen1994usability}
	\item \textit{Handbook of Usability Testing: How to Plan, Design, and Conduct Effective Tests} \cite{rubin2011}
	\item Avaliação de usabilidade de sites web \cite{winckler2022}
\end{itemize}

\section{Metodologia de Desenvolvimento}
\label{sec:Metodologia de Desenvolvimento}