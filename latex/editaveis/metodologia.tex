\chapter[Metodologia]{Metodologia}
\label{chap:Metodologia}

O objetivo deste capítulo é fornecer uma visão geral das metodologias utilizadas ao longo do trabalho nas etapas de escrita, exploração teórica e desenvolvimento prático. 
A estrutura do capítulo inicia-se pela \hyperref[sec:Classificacao da Pesquisa]{Classificação da Pesquisa Científica}, considerando \hyperref[sec:Objetivos2]{Objetivos}, \hyperref[sec:Natureza]{Natureza}, \hyperref[sec:Abordagem]{Abordagem} e \hyperref[sec:Procedimentos]{Procedimentos}. 
Em seguida, tem-se o \hyperref[sec:Fluxo de Atividades]{Fluxo de Atividades}, que guia o desenvolvimento do trabalho, o \hyperref[sec:Levantamento Bibliografico]{Levantamento Bibliográfico}, a \hyperref[sec:Metodologia de Desenvolvimento]{Metodologia de Desenvolvimento}, e a 
\hyperref[sec:Metodologia de Analise de Resultados]{Metodologia de Análise de Resultados}. Por fim, é definido o \hyperref[sec:Cronograma]{Cronograma de Atividades}, bem como o \hyperref[sec:Resumo do Capitulo]{Resumo do Capítulo}, que possui uma visão completa das etapas 
metodológicas abordadas.

\section{Classificação da Pesquisa}
\label{sec:Classificacao da Pesquisa}
De acordo com \citeonline{gil2002}, pode-se definir pesquisa como o procedimento racional e sistemático que tem por objetivo proporcionar respostas aos problemas que são propostos. É desenvolvida através da colaboração 
dos conhecimentos existentes e a aplicação cuidadosa de métodos, técnicas e outros procedimentos científicos.

Com relação às escolhas metodológicas, podem ser utilizadas classificações quanto ao objetivo da pesquisa; quanto à natureza da pesquisa; quanto à abordagem do problema, e quanto aos procedimentos serem realizados. O intuito desta seção é 
categorizar a investigação realizada neste estudo com base nesses critérios.

\subsection{Objetivos}
\label{sec:Objetivos2}
Segundo \citeonline{malhotra2001}, as pesquisas podem ser classificadas, em termos amplos, como exploratórias ou conclusivas. Sendo a pesquisa exploratória caracterizada por \citeonline{gil2002} como tendo como principal objetivo 
desenvolver, esclarecer e modificar conceitos e ideias, com a finalidade de formular problemas mais precisos ou hipóteses pesquisáveis para estudos futuros.

Visando explorar a usabilidade e a experiência de usuário de uma aplicação específica, com público alvo definido, pode-se classificar a presente pesquisa como Pesquisa Exploratória, com objetivo de proporcionar maior conhecimento 
sobre os tópicos em estudo.

Este tipo de pesquisa tem como objetivo conferir maior familiaridade com o problema, com vistas a torná-lo mais explícito ou a construir hipóteses. A grande maioria dessas pesquisas envolve: (a) levantamento 
bibliográfico; (b) 	entrevistas com pessoas que tiveram experiências práticas com o problema pesquisado; e (c) análise de exemplos que estimulem a compreensão \cite{gil2002}.

\subsection{Natureza}
\label{sec:Natureza}
\citeonline{gerhardt2009} propõem uma classificação do ponto de vista da natureza de pesquisa em dois grupos distintos:

\begin{itemize}
	\item Pesquisa Básica - objetiva gerar conhecimentos novos que sejam úteis para o avanço da ciência, sem uma aplicação prática prevista, e
	\item Pesquisa Aplicada - objetiva gerar conhecimentos direcionados à aplicação prática, com foco na resolução de problemas específicos, e 
	proporcionando insumos mais concretos sob o objeto em estudo.
\end{itemize}

Quanto à natureza, a pesquisa deste trabalho pode ser classificada como Aplicada, pois visa a aplicação prática de melhorias em uma aplicação específica, no caso, o Aplicativo Multilind.

\subsection{Abordagem}
\label{sec:Abordagem}
Do ponto de vista da abordagem do problema, pode-se classificar, de acordo com \citeonline{gil2002}, como sendo uma pesquisa quantitativa e qualitativa, com maior tendência ao viés qualitativo. A pesquisa quantitativa traduz opiniões e informações em 
números para classificá-las e analisá-las. Já a pesquisa qualitativa considera as subjetividades entre o sujeito e o mundo real, não sendo traduzida em números, e tendo como principais focos o processo e seu significado.

A abordagem dessa pesquisa pode ser classificada tanto como qualitativa, quanto como quantitativa. No aspecto qualitativo, são considerados critérios subjetivos, associados à experiência de usuário, com destaque à motivação/engajamento; à identificação 
do usuário com o aplicativo; aos sucesso e esforço ao utilizá-lo, dentre outros. As percepções sobre esses critérios são anotadas em escalas, tal como a escala Likert, que permite associar sentimentos de valor em uma pontuação. Por exemplo: 1 = desmotivado; 
4 = indiferente, e 7 = motivado. Por outro lado, no aspecto quantitativo, a análise da usabilidade envolve testes com métricas mensuráveis, tal como quanto tempo o usuário levou para executar um dado fluxo de relevância na interface do aplicativo, ou ainda 
quanto tempo o usuário ficou em uma dada tela, sem tomada de decisão alguma.

\subsection{Procedimentos}
\label{sec:Procedimentos}
Do ponto de vista dos procedimentos técnicos, \citeonline{gil2002} caracteriza a pesquisa da seguinte forma:

\begin{itemize}
	\item Pesquisa Bibliográfica: baseada em material já publicado, como livros, artigos de periódicos e conteúdo disponibilizado na Internet;
	\item Pesquisa Documental: utiliza materiais que não necessariamente passaram por análises mais criteriosas (ex. avaliação por pares), como documentos, arquivos e registros;
	\item Pesquisa Experimental: determina um objeto de estudo; seleciona variáveis que podem influenciá-lo, e estabelece formas de controle e observação dos efeitos produzidos pelas variáveis no objeto;
	\item Levantamento: envolve a coleta direta de informações por meio de questionários ou entrevistas, com o objetivo de conhecer o comportamento das pessoas;
	\item Estudo de caso: realiza uma análise detalhada e aprofundada de um ou poucos objetos, permitindo um amplo conhecimento deles;
	\item Pesquisa Expost-Facto: ocorre após os eventos acontecerem, realizando um "experimento" posterior aos fatos;
	\item Pesquisa-Ação: concebida e realizada em associação direta com uma ação ou resolução de um problema coletivo, com a participação colaborativa dos pesquisadores e dos participantes envolvidos na situação ou problema, e
	\item Pesquisa Participante: desenvolve-se por meio da interação entre pesquisadores e membros das situações investigadas.
\end{itemize}

Essa pesquisa pode ser classificada em três categorias: pesquisa bibliográfica, pesquisa-ação e estudo de caso. A pesquisa bibliográfica consiste em realizar levantamentos bibliográficos nas áreas de usabilidade e experiência do usuário, 
utilizando esses conhecimentos como base para a realização do estudo; a pesquisa-ação é um termo amplo que descreve um ciclo de melhorias, percorrendo os campos da prática e da investigação, geralmente envolvendo etapas de planejamento, 
implementação, descrição e avaliação \cite{tripp2005}, e o estudo de caso envolve a análise aprofundada de uma entidade específica, neste caso, o aplicativo Multilind, com o objetivo de compreender o funcionamento do contexto a partir da perspectiva 
dos usuários \cite{gerhardt2009}.

\section{Fluxo de Atividades}
\label{sec:Fluxo de Atividades}
A condução dessa monografia requer a definição de um fluxo de atividades estruturado em diversas etapas fundamentais. Essas etapas são representadas e organizadas no modelo ilustrado na Figura \ref{fig04}, seguindo a notação BPMN (\textit{Business Process Model and Notation}). 
No contexto do primeiro estágio da elaboração do TCC, têm-se as seguintes atividades e subprocessos:

\begin{itemize}
	\item Definir Tema: Atividade destinada à escolha do tema de desenvolvimento do trabalho. Com o auxílio da orientadora e visando uma das áreas de interesse da autora, foi definido o tema: Aplicativo Multilind: Melhoria Orientada à
	Experiência de Usuário e à Usabilidade. \textit{Status}: \textbf{Concluída};
	\item Contextualizar Problema: Atividade que consiste em desenvolver uma visão clara em relação ao contexto do aplicativo e o problema que visa resolver. \textit{Status}: \textbf{Concluída} - \hyperref[chap:Introducao]{Capítulo 1};
	\item Levantamento Bibliográfico: Subprocesso responsável pela consulta a bases científicas para investigar a literatura especializada sobre Usabilidade, Experiência de Usuário e seus desdobramentos. \textit{Status}: \textbf{Concluída} - \hyperref[chap:Referencial]{Capítulo 2};
	\item Definir Suporte Tecnológico: Atividade que visa à definição de ferramentas e tecnologias que apoiam a realização do trabalho. \textit{Status}: \textbf{Concluída} - \hyperref[chap:ReferencialTech]{Capítulo 3};
	\item Especificar Metodologia: Atividade que trata de aspectos metodológicos da  pesquisa, abordando metodologias de cunho investigativo, de desenvolvimento, e de análise de resultado. \textit{Status}: \textbf{Concluída} - \hyperref[chap:Metodologia]{Presente Capítulo};
	\item Coletar Informações à respeito de Usabilidade e UX do aplicativo: Atividade com o objetivo de levantar métricas para a realização de melhorias na Usabilidade e na Experiência de Usuário pesquisa, abordando metodologias de cunho investigativo, de desenvolvimento, e de análise de resultado. \textit{Status}: \textbf{Concluída} - \hyperref[sec:Primeiro Ciclo]{Capítulo 5 - Primeiro Ciclo de Testes};
	\item Desenvolver Protótipo baseado em Melhorias: Atividade de criação de protótipo de alta fidelidade levando em consideração as informações levantadas e a literatura. \textit{Status}: \textbf{Concluída} - \hyperref[sec:Melhorias Propostas]{Capítulo 5 - Melhorias Propostas};
	\item Determinar Proposta do Trabalho: Atividade de detalhamento da proposta, considerando documentação dos dados coletados antes e depois das melhorias na aplicação; prova de conceito, e afins. \textit{Status}: \hyperref[chap:Proposta]{Capítulo 5}.
	\item Revisar TCC 1: Atividade que abrange todos os aspectos do TCC, incluindo sua escrita e os artefatos produzidos. Essa atividade é realizada por meio de iterações, tanto da autora quanto da orientadora. \textit{Status}: \textbf{Concluída} -  Insumos Presentes nesse Documento e Repositório Associado, e
	\item Apresentar TCC 1: Atividade de apresentação da primeira parte da monografia para a banca avaliadora. \textit{Status}: \textbf{Pendente}.
\end{itemize}

Em relação às atividades e aos subprocessos do segundo estágio da elaboração do TCC, têm-se:

\begin{itemize}
	\item Corrigir Apontamentos da Banca: Atividade voltada ao refinamento da monografia com base nas considerações da banca avaliadora. \textit{Status}: \textbf{A Realizar};
	\item Desenvolvimento da Aplicação: Subprocesso voltado ao desenvolvimento das melhorias de Usabilidade e Experiência do Usuário. O detalhamento desse subprocesso está disponível na seção \hyperref[sec:Metodologia de Desenvolvimento]{4.4}. \textit{Status}: \textbf{A Realizar};
	\item Análise de Resultados: Subprocesso que consiste na coleta de métricas e realização de melhorias com base nelas. Seu detalhamento está explicitado na seção \hyperref[sec:Metodologia de Analise de Resultados]{4.5}. \textit{Status}: \textbf{A Realizar};
	\item Revisar TCC 2: Atividade voltada ao aperfeiçoamento da escrita da monografia de forma iterativa. \textit{Status}: \textbf{A Realizar}.
	\item Apresentar TCC 2: Atividade de apresentação dos resultados obtidos - na totalidade - para a banca avaliadora. \textit{Status}: \textbf{A Realizar}, e
\end{itemize}

\begin{figure}[h!]
	\centering
	\caption{Fluxo de Atividades - Primeira Etapa do TCC}
	\resizebox{1\textwidth}{!}{\includegraphics{figuras/tcc1.eps}}
	\begin{tablenotes}[flushleft]
		\centering
		\item \textit{Fonte:} Autora.
	  \end{tablenotes}
	\label{fig04}
\end{figure}

\begin{figure}[h!]
	\centering
	\caption{Fluxo de Atividades TCC 2 - Segunda Etapa do TCC}
	\resizebox{1\textwidth}{!}{\includegraphics{figuras/tcc2.eps}}
	\begin{tablenotes}[flushleft]
		\centering
		\item \textit{Fonte:} Autora.
	  \end{tablenotes}
	\label{fig05}
\end{figure}

\section{Levantamento Bibliográfico}
\label{sec:Levantamento Bibliografico}
De acordo com \citeonline{gil2002}, o levantamento bibliográfico preliminar é fundamental na elaboração da pesquisa, pois proporciona familiaridade com a área de estudo. No presente trabalho, , a investigação sob o tema de interesse ocorreu orientando-se por artigos, livros e outras monografias, a fim de solidificar sua base teórica.

Para garantir a qualidade e a atualidade das fontes utilizadas, as plataformas de base científica consultadas foram Periódicos Capes e Science Direct, que se destacam pela revisão em pares de seus artigos. Adicionalmente, foi utilizado o Google Scholar, o que permitiu uma abordagem ampla em termos investigativos, incluindo livros e outras publicações acadêmicas. 

\subsection{\textit{Strings} de Busca}
\label{sec:Strings de Busca}
Após a definição do tema, foram estabelecidas as \textit{strings} de busca para realizar pesquisas mais direcionadas nas principais bases de dados. A relação de busca por base de dados e seus resultados podem ser vistos na Tabela \ref{tab04}.

\begin{table}[h]
	\caption{\textit{Strings} de busca}
	\centering
	\label{tab04}
	\begin{tabular}{l|l|l}
	\hline
	\textit{String}               & Base de Dados            & Resultados \\ \hline
	'\textit{Software Usability}' & Periódico Capes          & 19.578     \\ \hline
	'\textit{User Experience}'    & Periódico Capes          & 159.636    \\ \hline
	'\textit{Usability Test}'     & Periódico Capes          & 15.082     \\ \hline
	'\textit{User Experience}'    & ScienceDirect (Elsevier) & 597.433    \\ \hline
	'\textit{Usability}'          & ScienceDirect (Elsevier) & 112        \\ \hline
	'\textit{Software Usability}' & Google Scholar           & 13.400     \\ \hline
	'\textit{User Experience}'    & Google Scholar           & 854.000    \\ \hline
	'\textit{Usability Test}'     & Google Scholar           & 39.300     \\ \hline
	\end{tabular}
	\begin{tablenotes}[flushleft]
		\centering
		\item \textit{Fonte:} Autora.
	\end{tablenotes}
\end{table}

\subsection{Critérios de Seleção}
\label{sec:Critérios de Selecao}
Durante a análise dos artigos, livros e periódicos relacionados aos temas descritos anteriormente, foram estabelecidos critérios para refinar os materiais. Os critérios de seleção utilizados para esse refinamento foram:

\begin{itemize}
	\item Ter sido escrito em português, inglês, alemão, e;
	\item Apontar métricas relacionadas à Experiência de Usuário e à Usabilidade.
\end{itemize}

Com base nisso, alguns dos principais artigos selecionados foram:

\begin{itemize}
	\item \textit{Usability is quality of use} \cite{bevan1995};
	\item \textit{User experience - a research agenda} \cite{hassenzahl2006};
	\item \textit{Attrakdiff: Ein fragebogen zur messung wahrgenommener hedonischer und pragmatischer qualität} \cite{hassenzahl2003};
	\item \textit{Usability Engineering} \cite{nielsen1994usability};
	\item \textit{Handbook of Usability Testing: How to Plan, Design, and Conduct Effective Tests} \cite{rubin2011};
	\item Avaliação de usabilidade de sites web \cite{winckler2022};
\end{itemize}

\section{Metodologia de Desenvolvimento}
\label{sec:Metodologia de Desenvolvimento}
Todas as atividades de desenvolvimento do projeto serão baseadas nos princípios ágeis, seguindo uma abordagem metodológica híbrida, que combina práticas do Scrum e do Kanban \cite{totvs2021} \cite{scrumguide2020}. 

De acordo com o Guia do Scrum, o Scrum é um \textit{framework} que pode ser utilizado por qualquer pessoa para lidar com problemas complexos, por meio de iterações adaptativas \cite{scrumguide2020}. Já o Kanban, permite 
a visualização do fluxo de trabalho em cada estágio e limitação do trabalho em progresso \cite{anderson2011}.

Sendo assim, a pesquisa seguirá o seguinte fluxo de desenvolvimento:

\begin{itemize}
	\item Elaborar o \textit{backlog}: permitindo estruturar as necessidades dos usuários, identificadas via Questionários, usando como base o aplicativo Multilind, em sua versão já exsistente. Essas possibilidades de melhorias para o aplicativo serão colocadas no estágio de \textit{Backlog};
	\item Obter o \textit{backlog} da \textit{sprint}:  identificando e selecionando histórias de usuário de relevância, com base na priorização conferida no Backlog do Projeto, visando o desenvolvimento das mesmas em uma sprint específica. Portanto, movendo essas histórias de usuário para o estágio de
	\textit{To Do}, em um Quadro Kanban;
	\item Desenvolver as histórias de usuário na \textit{sprint}: implementando as histórias de usuário selecionadas no \textit{backlog} da \textit{sprint}, e movendo-as para o estágio \textit{Doing}, em um Quadro Kanban;
	\item Conduzir processo de revisão das histórias de usuário implementadas: revisando o comportamento da aplicação, após o desenvolvimento pleno de uma história de usuário. Desenvolvimento esse realizado pela autora. Caso a tarefa seja
	aprovada, move-se a mesma para o estágio de \textit{Done}, no Quadro Kanban. Caso a tarefa não seja aprovada, retorna-se para a atividade anterior, e
	\item Proceder com a atualização do \textit{backlog} do produto: atualizando o \textit{backlog} do produto, considerando os status das tarefas - já realizada ou pendente. Caso ocorra a demanda por novas histórias de usuário, principalmente, em função dos \textit{feedbakcs} coletados via formulários no protocolo de pesquisa-ação, as mesmas serão incorporadas ao \textit{backlog} do produto
\end{itemize}

A figura X apresenta os detalhes abrangidos pelo subprocesso de Realização das Atividades de Desenvolvimento, que resume a Metodologia de Desenvolvimento.

\section{Metodologia de Análise de Resultados}
\label{sec:Metodologia de Analise de Resultados}
Com o objetivo de analisar os resultados obtidos no trabalho, será adotada a metodologia de pesquisa-ação. Essa abordagem consiste em investigar uma problemática e realizar uma ação ou resolução a partir dessa investigação, conforme descrito por \citeonline{gerhardt2009}. As etapas a 
serem seguidas são:

\begin{itemize}
	\item Captura de dados: condução de um levantamento de informações, realizado de forma \textit{online}, sendo assim assistido por recursos disponiveis na \textit{Web}, com uso de Questionários e ferramentas de testes de usabilidade. Trata-se de uma coleta classificada como ocasonal, uma vez que não ocorre de forma periódica (ou seja, com ocorrência estabelecida em periodicidade específica: de dois em dois meses), nem mesmo é realizada de forma contínua (ou seja, já faz parte de um processo estabelecido, sistemático e recorrente);
	\item Avaliação do plano de intervenção: os dados coletados serão analisados nesta etapa. A abordagem da pesquisa envolve tanto aspectos qualitativos quanto quantitativos. Os dados quantitativos serão analisados por meio de métricas concretas, como o número de cliques em um elemento 
	da interface gráfica. Já os dados qualitativos serão avaliados de forma mais criteriosa, utilizando-se questionários e mapas de calor, por exemplo.
	\item Desenvolvimento de um plano de ação: a partir da análise dos dados coletados nos formulários, será elaborado um plano de ação para resolver ou mitigar as insatisfações identificadas.
	\item Comunicação dos resultados: nesta etapa, os resultados e os levantamentos de cada ciclo de iteração da pesquisa-ação serão documentados de forma apropriada.
\end{itemize}

No primeiro ciclo da pesquisa-ação, o objetivo principal é avaliar a usabilidade e a experiência do usuário nos principais fluxos do aplicativo, utilizando um questionário baseado no \textit{AttrakDiff}. Com base na análise desses dados, será elaborada a primeira versão de melhoria no aplicativo Multilind.

No segundo ciclo, em que as melhorias já estarão projetadas, o objetivo é realizar medições utilizando o protótipo desenvolvido, a fim de verificar se os problemas identificados anteriormente foram mitigados ou solucionados. Os dados serão 
coletados de usuários que fazem parte do público-alvo.

Por fim, no terceiro ciclo, o objetivo é desenvolver de fato as melhorias, o que permitirá uma análise mais abrangente e possibilitará o registro dos resultados finais deste trabalho.

\section{Cronograma}
\label{sec:Cronograma}
Com base nos fluxos propostos anteriormente, foram elaborados os seguintes cronogramas para cumprir, respectivamente, a primeira etapa (Figura \ref{fig06}) e a segunda etapa (Figura \ref{fig07}) do TCC.

\begin{figure}[h!]
	\centering
	\caption{Cronograma - Primeira Etapa do TCC}
	\resizebox{1\textwidth}{!}{\includegraphics{figuras/tcc1s.eps}}
	\begin{tablenotes}[flushleft]
		\centering
		\item \textit{Fonte:} Autora.
	\end{tablenotes}
	\label{fig06}
\end{figure}

\begin{figure}[h!]
	\centering
	\caption{Cronograma - Segunda Etapa do TCC}
	\resizebox{1\textwidth}{!}{\includegraphics{figuras/tcc2s.eps}}
	\begin{tablenotes}[flushleft]
		\centering
		\item \textit{Fonte:} Autora.
	\end{tablenotes}
	\label{fig07}
\end{figure}

\section{Resumo do Capítulo}
\label{sec:Resumo do Capitulo}
O capítulo teve como objetivo apresentar os detalhes metodológicos a serem seguidos para a realização do trabalho. Inicialmente, a pesquisa foi classificada, em relação aos objetivos, como pesquisa exploratória, de natureza aplicada 
e de abordagem híbrida (quantitativa e qualitativa). Sob o ponto de vista de procedimentos técnicos, orienta-se por pesquisa bibliográfica, pesquisa-ação e estudo de caso. Em seguida, foi apresentado o \hyperref[sec:Fluxo de Atividades]{Fluxo de Atividades}
com as principais atividades e os subprocessos que envolvem a realização do trabalho, incluindo o \hyperref[sec:Levantamento Bibliografico]{Levantamento Bibliográfico}, a \hyperref[sec:Metodologia de Desenvolvimento]{Metodologia de Desenvolvimento}, combinando princípios de metodologias ágeis (Scrum \& Kanban); e a 
\hyperref[sec:Metodologia de Analise de Resultados]{Metodologia de Análise de Resultados}, , acordando sobre o protocolo de pesquisa-ação. Por fim, têm-se os \hyperref[sec:Cronograma]{Cronogramas} que possibilitam uma correspondência entre as atividades do trabalho e a variável tempo, tanto para o escopo da primeira 
etapa, quanto para o escopo da segunda etapa do TCC.