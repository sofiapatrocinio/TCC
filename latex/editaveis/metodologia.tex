\chapter[Metodologia]{Metodologia}
O objetivo deste capítulo é fornecer uma visão geral das metodologias utilizadas ao longo do trabalho nas etapas de escrita, exploração teórica e desenvolvimento prático. 
A estrutura do capítulo se inicia pela Metodologia da Pesquisa Científica, considerando Abordagem, Natureza, Objetivos e Procedimentos. Em seguida, tem-se o Fluxo de Atividades que guia o desenvolvimento  
da monografia, Metodologia de Desenvolvimento e de Análise de Resultados. Por fim, é definido o Cronograma de Atividades e tem-se o Resumo do Capítulo, que possui uma visão completa das etapas 
metodológicas abordadas.

\section{Classificação da Pesquisa}
\label{sec:Classificação da Pesquisa}
De acordo com Gil (1999), pode-se definir pesquisa como o procedimento racional e sistemático que tem por objetivo proporcionar respostas aos problemas que são propostos. É desenvolvida através da colaboração 
dos conhecimentos existentes e a aplicação cuidadosa de métodos, técnicas e outros procedimentos científicos.

Com relação as escolhas metodológicas, podem ser utilizadas classificações quanto ao objetivo da pesquisa, quanto a natureza da pesquisa e quanto a escolha do objeto de estudo. O objetivo desta seção é 
categorizar a investigação realizada neste estudo com base nesses critérios.

\subsection{Objetivos}
\label{sec:Objetivos}
