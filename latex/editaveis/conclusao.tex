\chapter[Conclusão]{Conclusão}
\label{chap:Conclusão}
Com o objetivo de encerrar este trabalho, este capítulo apresentará o \hyperref[sec:Contexto Geral]{Contexto Geral} (Seção \hyperref[sec:Contexto Geral]{7.1}) 
em que a pesquisa foi conduzida, fornecendo justificativas para sua execução. Além do \hyperref[sec:Status do Trabalho]{\textit{Status} 
do Trabalho} (Seção \hyperref[sec:Status do Trabalho]{7.2}) avaliando se os objetivos foram atingidos e respondendo à questão de pesquisa. Adicionalmente, 
serão destacadas as contribuições do projeto à sociedade, juntamente com suas principais limitações, em termos de 
\hyperref[sec:Contribuições e Limitações]{Contribuições e Limitações} (Seção \hyperref[sec:Contexto Geral]{7.3}). Além disso, as \hyperref[sec:Impressões da Autora]{Impressões da Autora} 
(Seção \hyperref[sec:Impressões da Autora]{7.4}) à respeito do projeto serão destacadas, bem como os aprendizados e desafios encontrados. Por fim, será apresentada a seção de 
\hyperref[sec:Trabalhos Futuros]{Trabalhos Futuros} (Seção \hyperref[sec:Trabalhos Futuros]{7.5}), 
que confere direções para a continuação do presente trabalho.

\section{Contexto Geral}
\label{sec:Contexto Geral}
A Década Internacional das Línguas Indígenas (DILI) foi estabelecida pela UNESCO em 2019, ao término do Ano Internacional das Línguas 
Indígenas. Essa iniciativa foi uma resposta à demanda dos povos indígenas da Bolívia, que reconheceram a importância de uma ação contínua em prol do reconhecimento, 
valorização e preservação das línguas indígenas em todo o mundo. 

A partir dessa demanda legítima, originada pelos povos indígenas bolivianos, a Declaração de Los Pinos foi elaborada no México em 2020, estabelecendo os fundamentos 
para a criação de um Plano de Ação Global para a DILI. Este documento enfatiza a participação efetiva dos povos indígenas em todas as etapas de decisão, consulta, 
planejamento e implementação, adotando como lema "Nada para nós sem nós" \cite{gtbrasil2024}.

Nesse contexto, em colaboração com a professora Altaci Corrêa Rubim\footnote{\url{https://amazoniareal.com.br/personagem/altaci-correa-rubim/}(último acesso: Março 2024)}, 
que é a representante brasileira do Grupo de Trabalho Mundial da Década das Línguas Indígenas, e membros do GT do Brasil, o Multilind foi desenvolvido. 

Este estudo abrangeu uma análise abrangente da versão atual do aplicativo, visando compreender tanto os aspectos positivos quanto negativos sob a perspectiva dos usuários. 
Além disso, incluiu pesquisas na literatura especializada sobre personas, usabilidade, experiência do usuário e testes.  Resultados desses levantamentos e estudo encontram-se 
nos seguintes capítulos: \hyperref[chap:Referencial]{Capítulo 2 - Referencial Teórico}, \hyperref[chap:ReferencialTech]{Capítulo 3 - Referencial Tecnológico}, e 
\hyperref[chap:Aplicativo Multilind]{Capítulo 5 - Aplicativo Multilind}.

Baseando-se na literatura utilizada e na metodologia de análise de resultados estabelecida, foi realizado um ciclo de pesquisa-ação envolvendo usuários que se enquadram nos 
perfis definidos como público-alvo do Multilind. Este ciclo incluiu atividades previamente definidas e coleta de métricas específicas, permitindo a avaliação da última versão 
do aplicativo Multilind por meio de testes de usabilidade e aplicação do Questionário \textit{Attrakdiff} em comparação com a versão com as melhorias propostas.

A partir deste estudo, é possível concluir que a aplicação da abordagem proposta para analisar a usabilidade e a experiência do usuário de aplicativos móveis pode fornecer 
resultados significativos sobre problemas e pontos de melhoria nesses aspectos. Além disso, foi possível perceber que os resultados dos testes sugerem que as melhorias 
implementadas tiveram um impacto positivo na usabilidade e experiência do usuário no aplicativo Multilind.

\section{\textit{Status} do Trabalho}
\label{sec:Status do Trabalho}
O principal objetivo deste trabalho, como definido na \hyperref[sec:Objetivos]{Seção 1.4 - Objetivos}, foi a melhoria do aplicativo Multilind, procurando aprimorá-lo em termos 
de experiência de usuário e orientando-se por boas práticas de usabilidade.

A partir da análise realizada, podemos afirmar que este objetivo foi alcançado, uma vez que os resultados dos testes sugerem que as melhorias implementadas tiveram um impacto 
positivo na usabilidade e na experiência do usuário no aplicativo Multilind.

Com o propósito de atender ao objetivo geral, foram definidos objetivos específicos, de menor escopo cada. Os objetivos específicos cumpridos estão listados a
seguir:

    \begin{itemize}
              \item Especificação clara do público alvo, sendo esse centrado em pessoas indígenas, evidenciando esse perfil de usuários com uso de personas;

              \item Levantamento de referenciais teóricos, no que tange aos conceitos de usabilidade e experiência de usuário;

              \item Detalhamento de tecnologias e outros recursos técnicos que sejam adequados para a melhoria do aplicativo Multilind;

              \item Uso de prototipação, visando validações pontuais junto aos interessados, facilitando a melhoria do aplicativo;

              \item Condução da análise de resultados no aplicativo já melhorado, e
              
              \item Documentação do trabalho como um todo, orientando-se por boas práticas da Engenharia de Software.
    \end{itemize}

A questão de pesquisa, definida na \hyperref[sec:QuestaodePesquisa]{Seção 1.3 - Questão de Pesquisa}, foi "É possível melhorar a experiência do usuário no aplicativo 
Multilind aplicando boas práticas de usabilidade?". Com base nos resultados do segundo ciclo de testes, é possível concluir que houve uma clara melhoria na compreensão da aplicação 
pelos usuários devido à inclusão de um passo a passo inicial e à validação das mudanças nos fluxos do aplicativo, tornando-os mais intuitivos. Os dados revelaram melhorias em todas 
as quatro principais dimensões do AttrakDiff, que avalia a experiência do usuário.

\section{Contribuições e Limitações}
\label{sec:Contribuições e Limitações}

As contribuições deste trabalho para a melhoria do aplicativo Multilind são significativas, uma vez que se empenharam em aprimorar a usabilidade e a experiência do usuário no aplicativo. 
Com a implementação de melhorias como a inclusão de um passo a passo inicial e a validação das mudanças nos fluxos do aplicativo, os usuários puderam notar maior facilidade de uso. 
Isso pode ser observado pela melhoria nas quatro principais dimensões do \textit{AttrakDiff}, que trata perspectivas hedônicas, emocionais e experienciais sobre a qualidade do produto na visão 
dos usuários.

Os aprimoramentos feitos no aplicativo Multilind não apenas promovem a valorização e o reconhecimento das línguas indígenas brasileiras, mas também facilitam a disseminação de informações sobre 
essas línguas. Ao tornar o aplicativo mais atrativo e eficaz para os usuários, ele se torna uma ferramenta que auxilia na promoção, difusão e vitalização das línguas indígenas, contribuindo com 
a preservação da diversidade linguística e cultural do Brasil.

Até o momento, não existe uma aplicação que aborde de forma abrangente tanto o mapeamento das línguas quanto os dicionários de diversas dessas línguas, dentre as quase duzentas existentes no Brasil. Embora 
existam diversas iniciativas voltadas para a preservação e documentação das línguas indígenas, muitas delas se concentram em aspectos específicos, como a coleta de dados linguísticos ou a criação de 
dicionários para línguas individuais. No entanto, uma aplicação que integre esses dois aspectos, oferecendo um amplo conjunto de informações sobre as línguas indígenas do Brasil, ainda não foi desenvolvida.

A complexidade e a diversidade linguística dos povos indígenas do Brasil representam um desafio único para a criação de uma aplicação abrangente. Cada língua possui suas próprias características, estruturas 
e nuances, o que requer uma abordagem cuidadosa e especializada para sua documentação e preservação. Nesse sentido, a parceria com o GT do Brasil, que propõe revisar os dados, garantindo a precisão e autenticidade 
das informações disponibilizadas, é fundamental. 

Essa colaboração entre os membros do GT, que incluem indígenas e linguistas especializados, permite uma revisão detalhada e contextualizada, levando em consideração não apenas aspectos linguísticos, mas também 
culturais e históricos dos povos indígenas, valorizando e preservando seus conhecimentos tradicionais.

\section{Impressões da Autora}
\label{sec:Impressões da Autora}



\section{Trabalhos Futuros}
\label{sec:Trabalhos Futuros}
